




\section{PRFs from DDH: Naor-Reingold PRF}
We will now describe a PRF function family $F_n: \mathcal{K} \times \{0,1\}^n \rightarrow \mathbb{G}_n$ where DDH is assumed to be hard for  $\{\mathbb{G}_n\}$ and $\mathcal{K}$ is the key space.
The seed for the PRF $F_n$ will be $K =  (h, u_1, \ldots u_n)$, where $u,u_0\ldots u_n$ are sampled uniformly from $|\mathbb{G}_n|$, $g$ is the generator of $\mathbb{G}_n$ and $h = g^u$. 

\[F_n(K,x) = h^{\prod_{i} u_i^{x_i}}\]

Next, we will prove that the function $F_n$ is a pseudo-random function or that $\{F_n\}$ is a pseudo-random function ensemble.\footnote{Here, we require that adversary distinguish the function $F_n$ from a random function from $\{0,1\}^n$ to $\mathbb{G}_n$. Note that the output range of the function is $\mathbb{G}_n$. Note that the distribution of random group elements in $\mathbb{G}_n$ might actually be far from uniformly random strings.}
\begin{lemma}
Assuming the DDH Assumption (see Definition~\ref{def:ddh}) for $\{\mathbb{G}_n\}$ is hard, we have that $\{F_n\}$ is a pseudorandom function ensemble.
\end{lemma}
\begin{proof}
The proof of this lemma is similar to the proof of Theorem~\ref{theorem:ggm}.

Let $R_n^j$ be random function from $\{0,1\}^j \rightarrow \mathbb{G}_n$. Then we want to prove that for all non-uniform PPT adversaries $\mathcal{A}$ we have that:
\[\mu(n) = \left|\Pr[\mathcal{A}^{F_n}(1^n) =1] -  \Pr[\mathcal{A}^{R_n^n}(1^n) =1]\right|\]
is a negligible function. 

For the sake of contradiction, we assume that the function $F_n$ is not pseudorandom. Next, towards a contradiction, we consider a sequence of hybrid functions $F_n^0 \ldots F_n^n$. 
For any $j \in \{0,\ldots n\}$, let $F^j_n((h,u_{j}\ldots u_n),x) = (R_n^j(x_1\ldots x_j))^{\prod_{i=j+1}^n u_i^{x_i}}$, where $R_n^0(\epsilon)$ is the constant function with output $h$. Observe that $F_n^0$ is the same as the function $F_n$ and $F_n^n$ is the same as the function $R_n^n$. Thus, by a hybrid argument, we conclude that there exists $k \in \{0,\ldots n-1\}$, such that 
\[\left|\Pr[\mathcal{A}^{F_n^k}(1^n) =1] -  \Pr[\mathcal{A}^{F_n^{k+1}}(1^n) =1]\right|\]
is a non-negligible function. Now all we are left to show is that this implies an attacker that refutes the DDH assumption. The proof of this claim follows by a sequence of $T$ sub-hybrids, where $T$ is the running time of $\mathcal{A}$. Without loss of generality we assume that $\mathcal{A}$ never makes the same query twice. 

More specifically, we consider a sequence of functions $F_n^{k,t}$ where $t \in \{0,T\}$, $F_n^{k,0}$ is same as $F_n^{k}$ and $F_n^{k,T}$ is same as $F_n^{k+1}$. In particular, we explain how $F_n^{k,t}$ answers queries by $\mathcal{A}$.\footnote{As assumed earlier, keep in mind that $\mathcal{A}$ never makes the same query twice.} Let $x^1, \ldots x^t$ be the first $t$ queries made by $\mathcal{A}$. For any query, $x$ made by $\mathcal{A}$ such that the first $k$ bits of $x$ match the first $k$ bits of one of $x_1, \ldots x_y$ answer as $F_n^{k+1}$ else answer as $F_n^{k}$. Now we can conclude that there exists a $t$ such that $F_n^{k,t}$ and $F_n^{k,t+1}$ are distinguishable with non-negligible probability. 

Finally, we will show that using an adversary that can distinguish between $F_n^{k,t}$ and $F_n^{k,t+1}$ we need to construct an adversary $\mathcal{B}$ that refutes the DDH assumption. We leave construction of this adversary as an exercise.
\end{proof}


\newpage
\section*{Exercises}
\begin{exercise}
\newcommand{\bit}{\{0,1\}}

Prove or disprove: If $f$ is a one-way function, then the following function $B:\bit^*\to\bit$ is a hardconcentrate predicate for $f$. The function $B(x)$ outputs the inner product modulo 2 of the first $\lfloor |x|/2\rfloor$ bits of $x$ and the last $\lfloor |x|/2\rfloor$ bits of $x$.
\end{exercise}

\begin{exercise}
Let $\phi(n)$ denote the first $n$ digits of $\pi = 3.141592653589\ldots$ after the decimal in binary ($\pi$ in its binary notation looks like $11.00100100001111110110101010001000100001\ldots$).

   Prove the following: if one-way functions exist, then there exists a one-way function $f$ such that the function $B:\{0,1\}^* \rightarrow \{0,1\}$ is not a hard concentrate bit of $f$. The function $B(x)$ outputs $\langle x, \phi(|x|)\rangle$, where
    \[\langle a, b\rangle := \sum_{i=1}^n a_i b_i \mod 2\]
    for the bit-representation of $a = {a_1a_2\cdots a_n}$ and $b= {b_1b_2\cdots b_n}$.
\end{exercise}

\begin{exercise}
 If $f: \{0,1\}^{n}\times \{0,1\}^n\rightarrow \{0,1\}^n$  is PRF, then in which of the following cases is $g: \{0,1\}^{n}\times \{0,1\}^n\rightarrow \{0,1\}^n$ also a PRF? \begin{enumerate} \item $g(K,x) = f(K,f(K,x))$ \item $g(K,x) = f(x,f(K,x))$ \item $g(K,x) = f(K,f(x,K))$
    \end{enumerate}
\end{exercise}

\begin{exercise}[Puncturable PRFs.] Puncturable PRFs are PRFs for which a key can be given out such that, it allows evaluation of the PRF on all inputs, except for one designated input.

\newcommand{\negl}{\mathsf{negl}}
\newcommand{\A}{\mathcal{A}}
\newcommand{\F}{F}
\newcommand{\KeyF}{\mathsf{Key}_{\F}}
\newcommand{\PunctureF}{\mathsf{Puncture}_{\F}}
\newcommand{\EvalF}{\mathsf{Eval}_{\F}}


A puncturable pseudo-random function $\F$ is given by a triple of efficient algorithms ($\KeyF$,$\PunctureF$, and $\EvalF$), satisfying the following conditions:
\begin{itemize}
\item[-] \textbf{Functionality preserved under puncturing}: For every $x^*, x \in \{0,1\}^{n}$ such that $x^* \neq x$, we have that:
    $$\Pr[\EvalF(K,x) = \EvalF(K_{x^*},x) : K \gets \KeyF(1^n), K_{x^*} = \PunctureF(K,x^*)] = 1$$
\item[-] \textbf{Pseudorandom at the punctured point}: For every $x^*\in \{0,1\}^n$ we have that for every polysize adversary $\A$ we have that:
    $$|\Pr[\A(K_{x^*}, \EvalF(K,x^*)) = 1] - \Pr[\A(K_{x^*}, \EvalF(K,U_n)) = 1]|= \negl(n)$$
    where $K \gets \KeyF(1^n)$ and $K_S = \PunctureF(K,x^*)$. $U_n$ denotes the uniform distribution over $n$ bits.
\end{itemize}

Prove that: If one-way functions exist, then there exists a puncturable PRF family that maps $n$ bits to $n$ bits. \\ 
\textbf{Hint:} The GGM tree-based construction of PRFs from a length doubling pseudorandom generator (discussed in class) can be adapted to construct a puncturable PRF. Also note that $K$ and $K_{x^*}$ need not be the same length.
\end{exercise}
%
%\subsection{Application}
%Consider an interesting game: Alice and Bob are talking on the phone.
%Alice flips a coin, and Bob guesses whether it's head or tail.
%But the problem is how can Alice convince Bob that the coin is indeed head or tail?
%If we have pseudorandom functions, the problem could be easily solved.
%
%Assume we have a PRF $F_n: \{0, 1\}^n \rightarrow \{0, 1\}^n$.
%Alice and Bob have a shared key $i \in \{0, 1\}^n$, then $f_i(\cdot)$ is shared information.
%Now Alice has a message $m \in \{0, 1\}^n$ and wants to let Bob guess it,
%the procedure consists of three steps.
%\begin{enumerate}[(a)]
%    \item Alice chooses a string $r \in \{0, 1\}^n$, and sends to Bob  $m' = f_i(r) \oplus m$ ;
%    \item Bob guesses $m$;
%    \item Alice sends $r$ to Bob.
%\end{enumerate}
%In step (a), since $F_n$ is PRF, all the information that Bob gets is a random $n$-bit string, so it will not influence his behavior in step (b).
%Then in step (c), Bob receives $r$ and will be convinced that the true value of $m$ is $f_i(r) \oplus m'$.
%