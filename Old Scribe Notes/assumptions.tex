\chapter{Number Theoretic Background and Constructions}
So far, much of our investigation relied on the existence of one-way-functions or in certain cases on the existence of one-one one-way functions. However, just the mere existence of an object is not enough for real-world implementations. In this chapter, we will define certain number theoretic problems that are conjectured to be hard. We will then be interested in making conjectures about specific functions being one-way.

\section{The Discrete-Log Family of Problem}
Consider a group $\mathbb{G}$ of prime order. For example, consider the group $\mathbb{Z}_p^*$ where $p$ is a large prime. Let $g$ be a generator of this group $\mathbb{G}$. In this group, given $g^x$ for a random $x \in \{1,\ldots p-1\}$ consider the problem of finding $x$. This problem, referred to as the discrete-log problem, is believed to be computationally hard.

As in the case one-way functions, asymptotic definition of the discrete-log problem needs to consider an infinite family of groups or what we will a group ensemble. 

\paragraph{Group Ensemble.} A group ensemble is a set of finite cyclic groups $\mathcal{G} =\{ \mathbb{G}_n\}_{n \in \mathbb{N}}$. For the group $G_n$, we assume that given two group elements in $G_n$, their sum can be computed in polynomial in $n$ time. Additionally, we assume that given $n$ the generator $g$ of $\mathbb{G}_n$ can be computed in polynomial time. 

\begin{definition}[Discrete-Log Assumption]
We say that the discrete-log assumption holds for the group ensemble $\mathcal{G} =\{ \mathbb{G}_n\}_{n \in \mathbb{N}}$, if for every non-uniform PPT algorithm $\mathcal{A}$ we have that
\[\mu_\mathcal{A}(n) := \Pr_{x \leftarrow |G_n|}[\mathcal{A}(g,g^x) = x]\]
is a negligible function.
\end{definition}

\paragraph{The Diffie-Hellman Problems.} In addition to the discrete-log assumption, we also define the Computational Diffie-Hellman Assumption and the Decisional Diffie-Hellman Assumption. 

\begin{definition}[Computational Diffie-Hellman (CDH) Assumption]
We say that the Computational Diffie-Hellman Assumption holds for the group ensemble $\mathcal{G} =\{ \mathbb{G}_n\}_{n \in \mathbb{N}}$, if for every non-uniform PPT algorithm $\mathcal{A}$ we have that
\[\mu_\mathcal{A}(n) := \Pr_{x,y \leftarrow |G_n|}[\mathcal{A}(g,g^x,g^y) = g^{xy}]\]
is a negligible function.
\end{definition}

\begin{definition}[Decisional Diffie-Hellman (DDH) Assumption]
We say that the Computational Diffie-Hellman Assumption holds for the group ensemble $\mathcal{G} =\{ \mathbb{G}_n\}_{n \in \mathbb{N}}$, if for every non-uniform PPT algorithm $\mathcal{A}$ we have that
\[\mu_\mathcal{A}(n) = \left\mid\Pr_{x,y \leftarrow |G_n|}[\mathcal{A}(g,g^x,g^y,g^{xy}) = 1] - \Pr_{x,y,z \leftarrow |G_n|}[\mathcal{A}(g,g^x,g^y,g^{z}) = 1]\right\mid\]
is a negligible function.
\end{definition}

It is not hard to observe that the discrete-log assumption is the weakest of the three assumptions above. In fact, it is not difficult to show that the Discrete-Log Assumption for $\mathcal{G}$ implies the CDH and the DDH Assumptions for $\mathcal{G}$.  Additionally, we leave it as an exercise to show that the CDH Assumption for $\mathcal{G}$ implies the  DDH Assumptions for $\mathcal{G}$.

\paragraph{Examples of Groups where these assumptions hold.} Now we provide some examples of group where these assumptions hold. 
\begin{enumerate}
    \item Consider the group $\mathbb{Z}_p^*$ for a prime $p$.\footnote{Since the number of primes is infinite we can define an infinite family of such groups. For the sake of simplicity, here we only consider a single group.} For this group the CDH Assumption is conjectured to be true. However, using the Legendre symbol,\footnote{Let $p$ be an odd prime number. An integer $a$ is said to be a \emph{quadratic residue} modulo $p$ if it is congruent to a perfect square modulo $p$ and  is said to be a \emph{quadratic non-residue} modulo $p$ otherwise. The \emph{Legendre symbol} is a function of $a$ and $p$ defined as
    \begin{equation*}
        \left(\frac{a}{p}\right) = \begin{cases}
    1 &\text{if $a$ is quadration residue mod $p $ and $a \not\equiv 0 \mod p$}\\
    -1 &\text{if $a$ is quadration non-residue mod $p $}\\
    0 &\text{if $a \equiv 0 \mod p $}
    \end{cases}
    \end{equation*}
    
Legendre symbol can be efficiently computed as $\left(\frac{a}{p}\right) = a^{\frac{p-1}{2}}\mod p$.} 
the DDH Assumption in this group can be shown to be false. Can you show how?\footnote{This is because given $g^x, g^y$ one can easily compute deduce the Legendre symbol of $g^{xy}$.  Observe that if $\left(\frac{g}{p}\right) = -1$ then we have that $\left(\frac{g^{xy}}{p}\right) = 1$ if and only if $ \left(\frac{g^x}{p}\right) =1 $ or $\left(\frac{g^y}{p}\right) = 1$. Using this fact, we can construct an adversary that breaks the DDH problem with a non-negligible (in fact, noticeable) probability.}
\item Let $p = 2q+1$ where both $p$ and $q$ are prime.\footnote{By Dirichet's Theorem on primes in arithmetic progression, we have that there are infinite choices of primes $(p,q)$ for which $p = 2q+1$. This allows us to generalize this group to a group ensemble.} Next, let $\mathbb{Q}$ be the order-$q$ subgroup of quadratic residues in $\mathbb{Z}^*_p$. For this group, the DDH assumption is believed to hold. 
\item Let $N = pq$ where $p,q, \frac{p-1}{2}$ and $\frac{q-1}{2}$ are primes. Let $\mathbb{QR}_N$ be the cyclic subgroup of qudratic resides of order $\phi(N) = (p-1)(q-1)$. For this group $\mathbb{QR}_N$, the DDH assumption is also believed to hold.
%\item Pairings:
\end{enumerate}

\paragraph{Is DDH strictly stronger than Discrete-Log?} In the example cases above, where DDH is known believed to be hard, the best known algorithms for DDH are no better than the best known algorithms for the discrete-log problem. Whether the DDH assumption is strictly stronger than the discrete-log assumption is an open problem. For the best attacks on ``DDH-like'' assumptions see~\cite{EC:CorKog18}.



\section{CDH in $\mathbb{QR}_N$ implies Factoring}

In this section, we will show that the CDH assumption in $\mathbb{QR}_N$ implies the factoring assumption. 
\begin{lemma}
Given an algorithm $\mathcal{A}$ that breaks the CDH assumption in $\mathbb{T}_N$, we construct an non-uniform PPT adversary $\mathcal{B}$ that on input $N$ outputs its prime factors $p$ and $q$.
\end{lemma}
\begin{proof}
Given that $\mathcal{A}$ is an algorithm that solves the CDH problem in $\mathbb{QR}_N$ with a non-negligible probability, we construct an algorithm $\mathcal{B}$ that can factor $N$. Specifically, $\mathcal{B}$ on input $N$ proceeds as follows:
\begin{enumerate}
\item Sample $v \leftarrow \mathbb{QR}_N$ (such a $v$ can be obtained by sampling a random value in $\mathbb{Z}_N^*$ and squaring it) and compute $g := v^2 \mod N$.
\item Sample $x, y \leftarrow [N]$.\footnote{Note that sampling $x,y$ uniformly from $[N]$ is statistically close to sampling $x,y$ uniformly from $[\phi(N)]$.}
\item Let $ u := \mathcal{A}(g, g^{x}\cdot v, g^y\cdot v)$\footnote{Note that $g^x\cdot v$ where $x \leftarrow [N]$ is statistically close to $g^x$ where $x \leftarrow [N]$.} and compute $w := \frac{u}{g^{xy}\cdot v^{x+y}}$.
\item If $w^2 = v^2 \mod N$ and $u \neq \pm v$, then compute the factors of $N$ as $\mathsf{gcd}(N,u+v)$ and $N/\mathsf{gcd}(N,u+v)$. Otherwise, output $\bot$.
\end{enumerate}
Observe that if $\mathcal{A}$ solves the CDH problem then the returned values $u = g^{(x+ 2^{-1})(y + 2^{-1})} = v^{2xy + x+ y + 2^{-1}}$. Consequently, the computed value $w = v^{2^{-1}}$. Furthermore, with probability $\frac12$ we have that $w \neq v$. In this case, $\mathcal{B}$ can factor $N$.
\end{proof}

\section{OWFs from Discrete-Log}
Let's suppose that the discrete-log assumption hold for group ensemble $\mathcal{G} = \{\mathbb{G}_n\}$ then we have that the function family $\{f_n\}$ where $f_n: \{1,\ldots |\mathbb{G}_n|\}\rightarrow \mathbb{G}_n$ is a one-way function family. In particular, $f_n(x) = g^x$ where $g$ is the generator of the group $\mathbb{G}_n$. The proof that $\{f_n\}$ is one-way is left as as an exercise. 


\section{PRFs from DDH: Naor-Reingold PRF}
We will now describe a PRF function family $F_n: \mathcal{K} \times \{0,1\}^n \rightarrow \mathbb{G}_n$ where DDH is assumed to be hard for  $\{\mathbb{G}_n\}$ and $\mathcal{K}$ is the key space.
The seed for the PRF $F_n$ will be $K =  (h, u_1, \ldots u_n)$, where $u,u_0\ldots u_n$ are sampled uniformly from $|\mathbb{G}_n|$, $g$ is the generator of $\mathbb{G}_n$ and $h = g^u$. 

\[F_n(K,x) = h^{\prod_{i} u_i^{x_i}}\]

Next, we will prove that the function $F_n$ is a pseudo-random function or that $\{F_n\}$ is a pseudo-random function ensemble.\footnote{Here, we require that adversary distinguish the function $F_n$ from a random function from $\{0,1\}^n$ to $\mathbb{G}_n$. Note that the output range of the function is $\mathbb{G}_n$. Note that the distribution of random group elements in $\mathbb{G}_n$ might actually be far from uniformly random strings.}
\begin{lemma}
Assuming the DDH problem for $\{\mathbb{G}_n\}$ is hard, we have that $\{F_n\}$ is a pseudorandom function ensemble.
\end{lemma}
\begin{proof}
The proof of this lemma is similar to the proof of Theorem~\ref{theorem:ggm}.

Let $R_n^j$ be random function from $\{0,1\}^j \rightarrow \mathbb{G}_n$. Then we want to prove that for all non-uniform PPT adversaries $\mathcal{A}$ we have that:
\[\mu(n) = \left|\Pr[\mathcal{A}^{F_n}(1^n) =1] -  \Pr[\mathcal{A}^{R_n^n}(1^n) =1]\right|\]
is a negligible function. 

For the sake of contradiction, we assume that the function $F_n$ is not pseudorandom. Next, towards a contradiction, we consider a sequence of hybrid functions $F_n^0 \ldots F_n^n$. 
For any $j \in \{0,\ldots n\}$, let $F^j_n((h,u_{j}\ldots u_n),x) = (R_n^j(x_1\ldots x_j))^{\prod_{i=j+1}^n u_i^{x_i}}$, where $R_n^0(\epsilon)$ is the constant function with output $h$. Observe that $F_n^0$ is the same as the function $F_n$ and $F_n^n$ is the same as the function $R_n^n$. Thus, by a hybrid argument, we conclude that there exists $k \in \{0,\ldots n-1\}$, such that 
\[\left|\Pr[\mathcal{A}^{F_n^k}(1^n) =1] -  \Pr[\mathcal{A}^{F_n^{k+1}}(1^n) =1]\right|\]
is a non-negligible function. Now all we are left to show is that this implies an attacker that refutes the DDH assumption. The proof of this claim follows by a sequence of $T$ sub-hybrids, where $T$ is the running time of $\mathcal{A}$. Without loss of generality we assume that $\mathcal{A}$ never makes the same query twice. 

More specifically, we consider a sequence of functions $F_n^{k,t}$ where $t \in \{0,T\}$, $F_n^{k,0}$ is same as $F_n^{k}$ and $F_n^{k,T}$ is same as $F_n^{k+1}$. In particular, we explain how $F_n^{k,t}$ answers queries by $\mathcal{A}$.\footnote{As assumed earlier, keep in mind that $\mathcal{A}$ never makes the same query twice.} Let $x^1, \ldots x^t$ be the first $t$ queries made by $\mathcal{A}$. For any query, $x$ made by $\mathcal{A}$ such that the first $k$ bits of $x$ match the first $k$ bits of one of $x_1, \ldots x_y$ answer as $F_n^{k+1}$ else answer as $F_n^{k}$. Now we can conclude that there exists a $t$ such that $F_n^{k,t}$ and $F_n^{k,t+1}$ are distinguishable with non-negligible probability. 

Finally, we will show that using an adversary that can distinguish between $F_n^{k,t}$ and $F_n^{k,t+1}$ we need to construct an adversary $\mathcal{B}$ that refutes the DDH assumption. We leave construction of this adversary as an exercise.
\end{proof}