\newcommand{\cO}{\ensuremath{\mathcal{O}}}
%\newcommand{\NC}[1]{\ensuremath{\mathbf{NC}^{#1}}}

\section{$\iO$ for Polynomial-sized Circuits}


\begin{definition}[Indistinguishability Obfuscator for $NC^1$]
Let $\Ck$ be the collection of circuits of size $O(\kappa)$ and depth
$O(\log{\kappa})$ with respect to gates of bounded fan-in.
Then a uniform PPT machine $\iO_{\NC{1}}$ is an
\emph{indistinguishability obfuscator} for circuit class $\NC{1}$ if it
is an indistinguishability obfuscator for $\Ck$.
\end{definition}

Given an indistinguishability obfuscator $\iO_{\NC{1}}$ for circuit
class $\NC{1}$, we shall demonstrate how to achieve an
indistinguishability obfuscator $\iO$ for all polynomial-sized circuits.
The amplification relies on fully homomorphic encryption (FHE).

\newcommand{\GEN}{\ensuremath{\mathsf{Gen}}}
%\newcommand{\Enc}{\ensuremath{\mathsf{Enc}}}
%\newcommand{\Dec}{\ensuremath{\mathsf{Dec}}}
%\newcommand{\Eval}{\ensuremath{\mathsf{Eval}}}
%\newcommand{\pk}{\ensuremath{\mathsf{pk}}}
%\newcommand{\sk}{\ensuremath{\mathsf{sk}}}

\begin{definition}[Homomorphic Encryption]
A \emph{homomorphic encryption scheme} is a tuple of PPT algorithms
$(\GEN, \Enc, \Dec, \Eval)$ as follows:
\begin{itemize}
\item
	$(\GEN, \Enc, \Dec)$ is a semantically-secure public-key
	encryption scheme.
\item
	$\Eval(\pk, C, e)$ takes public key $\pk$, an arithmetic circuit
	$C$, and ciphertext $e = \Enc(\pk, x)$ of some circuit input
	$x$, and outputs $\Enc(\pk, C(x))$.
\end{itemize}
\end{definition}

As an example, the ElGamal encryption scheme is homomorphic over the multiplication function.
Consider a cyclic group $G$ of order $q$ and generator $g$, and let
$\sk = a$ and $\pk = g^a$.
For ciphertexts $\Enc(\pk, m_1) = (g^{r_1}, g^{a r_1} \cdot m_1)$
and $\Enc(\pk, m_2) = (g^{r_2}, g^{a r_2} \cdot m_2)$, observe that
\begin{equation*}
\Enc(\pk, m_1) \cdot \Enc(\pk, m_2) = (g^{r_1 + r_2}, g^{a (r_1 + r_2)}
\cdot m_1 \cdot m_2) = \Enc(\pk, m_1 \cdot m_2)
\end{equation*}
Note that this scheme becomes additively homomorphic by encrypting $g^m$
instead of $m$.

\begin{definition}[Fully Homomorphic Encryption]
An encryption scheme is \emph{fully homomorphic} if it is both compact
and homomorphic for the class of all arithmetic circuits.
Compactness requires that the size of the output of $\Eval(\cdot, \cdot,
\cdot)$ is at most polynomial in the security parameter $\kappa$.
\end{definition}

\subsection{Construction}

%We first present a simpler construction under the virtual black box
%model, assuming the existence of a circuit obfuscator $\cO_{\NC{1}}$ for
%$\NC{1}$.

\newcommand{\prog}[1]{\ensuremath{P_{\pk_1,\pk_2,\sk_{#1},e_1,e_2}}}

Let $(\GEN, \Enc, \Dec, \Eval)$ be a fully homomorphic encryption
scheme.
We require that $\Dec$ be realizable by a circuit in $\NC{1}$.
The obfuscation procedure accepts a security parameter $\kappa$ and
a circuit $C$ whose size is at most polynomial in $\kappa$.
\begin{enumerate}
\item
	Generate $(\pk_1, \sk_1) \gets \GEN(1^\kappa)$ and
	$(\pk_2, \sk_2) \gets \GEN(1^\kappa)$.
\item
	Encrypt $C$, encoded in canonical form, as
	$e_1 \gets \Enc(\pk_1, C)$ and $e_2 \gets \Enc(\pk_2, C)$.
\item
	Output an obfuscation
	$\sigma = (\iO_{\NC{1}}(P), \pk_1, \pk_2, e_1, e_2)$
	of program $\prog{1}$ as described below.
\end{enumerate}

The evaluation procedure accepts the obfuscation $\sigma$ and program
input $x$.
\begin{enumerate}
\item
	Let $U$ be a universal circuit that computes $C(x)$ given a
	circuit description $C$ and input $x$, and denote by $U_x$ the
	circuit $U(\cdot, x)$ where $x$ is hard-wired.
	Let $R_1$ and $R_2$ be the circuits which compute
	$f_1 \gets \Eval(U_x, e_1)$ and $f_2 \gets \Eval(U_x, e_2)$,
	respectively.

\item
	Denote by $\omega_1$ and $\omega_2$ the set of all wires in $R_1$
	and $R_2$, respectively.
	Compute $\pi_1 : \omega_1 \to \{ 0, 1 \}$ and
	$\pi_2 : \omega_2 \to \{ 0, 1 \}$, which yield the value of internal
	wire $w \in \omega_1, \omega_2$ when applying $x$ as the input
	to $R_1$ and $R_2$.

\item
	Output the result of running $\prog{1}(x, f_1, \pi_1, f_2, \pi_2)$.
\end{enumerate}

Program $\prog{1}$ has $\pk_1$, $\pk_2$, $\sk_1$, $e_1$, and $e_2$
embedded.
\begin{enumerate}
\item
	Check whether $R_1(x) = f_1 \land R_2(x) = f_2$.
	$\pi_1$ and $\pi_2$ enable this check in logarithmic depth.
\item
	If the check succeeds, output $\Dec(\sk_1, f_1)$;
	otherwise output $\bot$.
\end{enumerate}

The use of two key pairs and two encryptions of $C$, similar to
CCA1-secure schemes seen previously, eliminates the virtual black-box
requirement for concealing $\sk_1$ within $\iO_{\NC{1}}(\prog{1})$.

\subsection{Proof of Security}

We prove the indistinguishability property for this construction
through a hybrid argument.

\newcommand{\Hyb}[1]{\ensuremath{\mathsf{H_{#1}}}}
\begin{proof}
Through the sequence of hybrids, we gradually transform an obfuscation
of circuit $C_1$ into an obfuscation of circuit $C_2$, with each
successor being indistinguishable from its antecedent.
\begin{description}
\item[$\Hyb{0}$]:
	This corresponds to an honest execution of $\iO(C_1)$.
	Recall that $e_1 = \Enc(\pk_1, C_1)$, $e_2 = \Enc(\pk_2, C_1)$,
	and $\sigma = (\iO_{\NC{1}}(\prog{1}), \ldots)$.

\item[$\Hyb{1}$]:
	We instead generate $e_2 = \Enc(\pk_2, C_2)$, relying on the
	semantic security of the underlying fully homomorphic encryption
	scheme.

\item[$\Hyb{2}$]:
	We alter program $\prog{2}$ such that it instead embeds $\sk_2$
	and outputs $\Dec(\sk_2, f_2)$.
	The output of the obfuscation procedure becomes
	$\sigma = (\iO_{\NC{1}}(\prog{2}, \ldots)$;
	we rely on the properties of functional equivalence and
	indistinguishability of $\iO_{\NC{1}}$.

\item[$\Hyb{3}$]:
	We generate $e_1 = \Enc(\pk_1, C_1)$ since $\sk_1$ is now
	unused, relying again on the semantic security of the fully
	homomorphic encryption scheme.

\item[$\Hyb{4}$]:
	We revert to the original program $\prog{1}$ and arrive
	at an honest execution of $\iO(C_1)$.

\end{description}
\end{proof}


\section{Identity-Based Encryption}

Another use of indistinguishability obfuscation is to realize
identity-based encryption (IBE).

\newcommand{\SETUP}{\ensuremath{\mathsf{Setup}}}
\newcommand{\KEYGEN}{\ensuremath{\mathsf{KeyGen}}}
\newcommand{\mpk}{\ensuremath{\mathsf{mpk}}}
\newcommand{\msk}{\ensuremath{\mathsf{msk}}}
\newcommand{\id}{\ensuremath{\mathsf{id}}}

\begin{definition}[Identity-Based Encryption]
An \emph{identity-based encryption scheme} is a tuple of PPT algorithms
$(\SETUP, \KEYGEN, \Enc, \Dec)$ as follows:
\begin{itemize}
\item
	$\SETUP(1^\kappa)$ generates and outputs a master public/private
	key pair $(\mpk, \msk)$.
\item
	$\KEYGEN(\msk, \id)$ derives and outputs a secret key
	$\sk_{\id}$ for identity $\id$.
\item
	$\Enc(\mpk, \id, m)$ encrypts message $m$ under identity $\id$
	and outputs the ciphertext.
\item
	$\Dec(\sk_{\id}, c)$ decrypts ciphertext $c$ and outputs the
	corresponding message if $c$ is a valid encryption under
	identity $\id$, or $\bot$ otherwise.
\end{itemize}
\end{definition}

\newcommand{\SIGN}{\ensuremath{\mathsf{Sign}}}
\newcommand{\VERIFY}{\ensuremath{\mathsf{Verify}}}

We combine an indistinguishability obfuscator $\iO$ with a digital
signature scheme $(\GEN, \SIGN, \VERIFY)$.
\begin{itemize}
\item
	Let $\SETUP \equiv \GEN$ and $\KEYGEN \equiv \SIGN$.
\item
	$\Enc$ outputs $\iO(P_m)$, where $P_m$ is a program that
	outputs (embedded) message $m$ if input $\sk$ is a secret key for
	the given $\id$, or $\bot$ otherwise.
\item
	$\Dec$ outputs the result of $c(\sk_{\id})$.
\end{itemize}
However, this requires that we have encryption scheme where the
``signatures'' do not exist.
We therefore investigate an alternative scheme.
%
%\newcommand{\Com}{\ensuremath{\mathsf{Com}}}
%
Let $(K, P, V)$ be a non-interactive zero-knowledge (NIZK) proof system.
Denote by $\Com(\cdot ; r)$ the commitment algorithm of a non-interactive
commitment scheme with explicit random coin $r$.

\begin{itemize}
\item
	Let $\sigma$ be a common random string.
	$\SETUP(1^\kappa)$ outputs $(\mpk = (\sigma, c_1, c_2), \msk =
	r_1)$, where $c_1 = \Com(0 ; r_1)$ and
	$c_2 = \Com(0^{|\id|} ; r_2)$.

\item
	$\KEYGEN(\msk, \id)$ produces a proof
	$\pi = P(\sigma, x_{\id}, s)$ for the following language $L$:
	$x \in L$ if there exists $s$ such that
\begin{equation*}
\underbrace{c_1 = \Com(0 ; s)}_{\text{Type I witness}} \lor
\underbrace{\left( c_2 = \Com(\id^* ; s) \land \id^* \ne \id
	\right)}_{\text{Type II witness}}
\end{equation*}

\item
	Let $P_{\id,m}$ be a program which outputs $m$ if
	$V(\sigma, x_{\id}, \pi_{\id}) = 1$ or outputs $\bot$ otherwise.

	$\Enc(\mpk, \id, m)$ outputs $\iO(P_{\id,m})$.
\end{itemize}

We briefly sketch the hybrid argument:
\begin{description}
\item[$\Hyb{0}$]:
	This corresponds to an honest execution as described above.
\item[$\Hyb{1}$]:
	We let $c_2 = \Com(\id^* ; r_2)$, relying on the hiding property
	of the commitment scheme.
\item[$\Hyb{2}$]:
	We switch to the Type II witness using
	$\pi_{\id_i} \forall i \in [q]$, corresponding to the queries
	issued by the adversary during the first phase of the
	selective-identity security game.
\item[$\Hyb{3}$]:
	We let $c_1 = \Com(1 ; r_1)$.
\end{description}

%\nocite{*}
%\printbibliography



% !TEX root = collection.tex

\newcommand{\extline}{$\scriptsize$-$\normalsize$\!}
\newcommand{\lextlineend}{$\scriptsize$\lhd\!$\normalsize$}
\newcommand{\rextlineend}{$\scriptsize\rule{.1ex}{0ex}$\rhd$\normalsize$}

\newcounter{index}

\newcommand\extlines[1]{%
  \setcounter{index}{0}%
  \whiledo {\value{index}< #1}
  {\addtocounter{index}{1}\extline}
}

\newcommand\rextlinearrow[2]{$
  \setbox0\hbox{$\extlines{#2}\rextlineend$}%
  \tiny$%
  \!\!\!\!\begin{array}{c}%
  \mathrm{#1}\\%
  \usebox0%
  \end{array}%
  $\normalsize$\!\!%
}

\newcommand\lextlinearrow[2]{$
  \setbox0\hbox{$\lextlineend\extlines{#2}$}%
  \tiny%
  $%
  \!\!\!\!\begin{array}{c}%
  \mathrm{#1}\\%
  \usebox0%
  \end{array}%
  $\normalsize$\!\!%
}

\renewcommand\lextlinearrow[2]{%
}

\renewcommand\rextlinearrow[2]{%
}
\renewcommand\lextlinearrow[2]{%
%  \setbox0\hbox{$\lextlineend\extlines{#2}$}%
   $\stackrel{\mathrm{#1}}{\leftarrow}$%
}

\renewcommand\rextlinearrow[2]{%
  %\setbox0\hbox{$\extlines{#2}\rextlineend$}%
  $\stackrel{\mathrm{#1}}{\rightarrow}$%
}



%\section{Indistinguishable Obfuscation Constructions using Puncturing}
\section{Digital Signature Scheme via Indistinguishable Obfuscation}
A digital signature scheme can be constructed via indistinguishable obfuscation (iO).  A digital signature scheme is made up of $(\SETUP, \SIGN, \VERIFY)$.\\

%\newcommand{\vk}{\mathsf{vk}}

\noindent $(\vk, \sk) \leftarrow \SETUP(1^k)$:\\
\indent $\sk$ = key of puncturable function and the seed of the PRF $F_k$\\
\indent $\vk = iO(P_k)$ where $P_k$ is the program:\\
\indent \indent $P_k(m, \sigma)$:\\
\indent \indent \indent for some OWF function $f$\\
\indent \indent \indent \indent return 1 if $f(\sigma) = f(F_k(m))$\\
\indent \indent \indent \indent return 0 otherwise\\

\noindent $\sigma \leftarrow \SIGN(\sk, m)$: Output $F_k(m)$.\\

\noindent $\VERIFY(\vk, m, \sigma)$: Output $P_k(m, \sigma)$.\\

\noindent Our security requirements will be that the adversary does wins the following game negligibly:\\

\begin{tabular}{llc}
{\large Challenger} & & {\large Adversary}\\
$(\vk, \sk) = \SETUP(1^k)$ and&&\\
picks random $m$&&\\
& \rextlinearrow{P_{k},m}{46} &\\
& \lextlinearrow{\sigma, m^*}{46} &\\
& Adversary wins game if $\VERIFY(\vk, m^*, \sigma) = 1$&
\end{tabular}\\

\noindent To prove the security of this system, we use a hybrid argument.  $H_0$ is as above.

\noindent $H_1$: Adjust $\vk$ so that $\vk = iO(P_{k, m, \alpha})$ where $\alpha = F_k(m)$ and $P_{k, m, \alpha}$ is the program such that:\\
\indent $P_{k,m, \alpha}(m^*, \sigma)$:\\
\indent \indent for some OWF $f$\\
\indent \indent \indent if $m = m^*$:\\
\indent \indent \indent \indent if $f(\sigma) = f(\alpha)$ return 1\\
\indent \indent \indent \indent otherwise return 0\\
\indent \indent \indent else proceed as $P_{k}$ from before\\
\indent \indent \indent \indent if $f(\sigma) = f(F_k(m^*))$ return 1\\
\indent \indent \indent \indent otherwise return 0\\
\noindent Note that this program does not change its output for any value. This is indistinguishable from $H_0$  by indistinguishability obfuscation.\\

\noindent $H_2$: Adjust $\alpha$ so that it is a randomly sampled value. The indistinguishability of $H_2$ and $H_1$ follows from the security of PRG.  \\
\noindent $H_3$: Adjust the program such that instead of $\alpha$ it relies on some $\beta$ that is compared instead $f(\alpha)$ in the third line.\\

Any adversary that can break $H_3$ non-negligibly can break the OWF $f$ with at the value $\beta$.

\section{Public Key Encryption via Indistinguishable Obfuscation}
A public key encryption scheme can be constructed via indistinguishable obfuscation.  A public key encryption scheme is made up of $(Gen, Enc, Dec)$.  The PRG used below is a length doubling PRG.\\

\noindent $(\pk, \sk) \leftarrow Gen(1^k)$\\
\indent $\sk$ = key of puncturable function and the seed of the PRF $F_k$\\
\indent $\pk = iO(P_k)$ where $P_k$ is the program:\\
\indent \indent $P_k(m, r)$:\\
\indent \indent \indent $t = PRG(r)$\\
\indent \indent \indent Output $c = (t, F_k(t) \oplus m)$\\

\noindent $Enc(\pk, m)$: Sample $r$ and output $(\pk(m,r))$.\\

\noindent $Dec(\sk = k, c = (c_1, c_2))$: Output $F_k(\sk,c_1) \oplus c_2$.\\

\noindent Our security requirements will be that the adversary does wins the following game negligibly:\\

\begin{tabular}{llc}
{\large Challenger} & & {\large Adversary}\\
$(\pk, \sk) = Gen(1^k)$ and&&\\
Randomly sample $b$ from $\{0,1\}$ and&&\\
$c^* = Enc(\pk, b)$ and&&\\
& \rextlinearrow{P_{k},c^*}{26} &\\
& \lextlinearrow{b^*}{26} &\\
& Adversary wins game if $b=b^*$&
\end{tabular}\\

\noindent To prove the security of this system, we use a hybrid argument.  $H_0$ is as above.

\noindent $H_1$: Adjust $\pk$ so that $\pk = iO(P_{k, \alpha, t})$ where $\alpha = F_k(t)$ and $P_{k, \alpha, t}$ is the program such that:\\
\indent $P_{k, \alpha, t}(m, r)$:\\
\indent \indent $t^* = PRG(r)$\\
\indent \indent if $t^* = t$, output $(t^*, \alpha \oplus m)$\\
\indent \indent else output $(t^*, F_k(t^*) \oplus m)$\\

\noindent Note that this program does not change its output for any value. This is indistinguishable from $H_0$  by indistinguishability obfuscation.\\

\noindent $H_2$: Adjust $\alpha$ so that it is a randomly sampled value.\\
\noindent $H_3$: Adjust the program such that $t^*$ is randomly sampled and is not in the range of the PRG.\\

Any adversary that can win $H_3$ can guess a random value non-negligibly.\\

\section{Indistinguishable Obfuscation Construction from $NC^1$ $iO$}
A construction of indistinguishable obfuscation from $iO$ for circuits in $NC^1$ is as follows:\\
Let $P_{k,C}(x)$ be the circuit that outputs the garbled circuit $\widetilde{UC(C,x)}$ with randomness $F_k(x)$ which is a punctured (at $k$) PRF in $NC^1$\\
\indent Note that $UC(C,x)$ outputs $C(x)$ ($UC$ is the ``universal'' circuit)\\
$iO(C) \rightarrow $ sample $k$ randomly from $\{0,1\}^{|x|}$ and output $iO_{NC^1}(P_{k,C})$ padded to a length $l$\\

As before, we use a hybrid argument to show the security for $iO$.\\
\noindent $H_0$: $iO(C) = iO_{NC^1}(P_{k,C})$ as above.\\
\noindent $H_{final} = H_{2^n}$: $iO(\pk, c_2)$\\
\noindent $H_1 \cdots H_i$: Create a program $Q_{k, c_1, c_2, i}(x)$ and obfuscate it.\\
$Q_{k,c_1,c_2,i}(x)$:\\
\indent Sample $k$ randomly\\
\indent if $x \ge i$, return $P_{k,c_1}(x)$\\
\indent else , return $P_{k,c_2}(x)$\\
\noindent Note that $H_i$ and $H_{i+1}$ are indistinguishable for any value other than $x=i$.\\
\noindent $H_{i,1}$ (between $H_i$ and $H_{i+1}$): Create a program $Q_{k, c_1, c_2, i, \alpha}(x)$, where $\alpha = Q_{k, c_1, c_2, i}(x)$ and obfuscate it.\\
$Q_{k, c_1, c_2, i, \alpha}(x)$:\\
\indent Sample $k$ randomly\\
\indent if $x = i$, return $\alpha$\\
\indent else , return $Q_{k,c_1,c_2, i}(x)$\\

\noindent $H_{i,2}$: Replace $\alpha$ with a random $\beta$ using fresh coins\\
\noindent $H_{i,3}$: Create the $c_2(x)$ value using fresh coins\\
\noindent $H_{i,4}$: Create the $c_2(x)$ value using $F_k(x)$\\
\noindent $H_{i,5}$: Finish the migration to $Q_{k,c_1,c_2,i+1}$\\

Note that at $H_{final}$, the circuit being obfuscated is completely changed from $c_1$ to $c_2$.




