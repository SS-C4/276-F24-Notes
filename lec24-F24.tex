\newcommand{\proofsketch}{\smallskip\noindent{\bf Proof sketch. }}
\algrenewcommand\algorithmicfunction{\textbf{Machine}}
\newcommand{\bits}{\set{0,1}}
\newcommand{\Ex}{\mathbb{E}}

\renewcommand{\O}{\ensuremath{\mathcal{O}}}
\newcommand{\To}{\rightarrow}
\newcommand{\e}{\epsilon}
\newcommand{\R}{\mathbb{R}}
\newcommand{\N}{\mathbb{N}}
\newcommand{\Z}{\mathbb{Z}}
\newcommand{\logAnd}{\wedge}

\newcommand{\indis}{\mathrel{\overset{\makebox[0pt]{\mbox{\normalfont\tiny c}}}{\approx}}}
\newcommand{\allindis}{\mathrel{\overset{\makebox[0pt]{\mbox{\normalfont\tiny p/s/c}}}{\approx}}}

\newcommand{\cclass}[1]{\mathsf{#1}}
\renewcommand{\P}{\cclass{P}}
\newcommand{\NP}{\cclass{NP}}
\newcommand{\Time}{\cclass{Time}}
\newcommand{\BPP}{\cclass{BPP}}
\newcommand{\Size}{\cclass{Size}}
\newcommand{\Ppoly}{\cclass{P_{/poly}}}
\newcommand{\CSAT}{\ensuremath{\mathsf{CSAT}}}
\newcommand{\SAT}{\ensuremath{\mathsf{3SAT}}}
\newcommand{\IS}{\mathsf{INDSET}}



\newcommand{\inp}{\mathsf{in}}
\newcommand{\outp}{\mathsf{out}}

\newcommand{\Param}{\kappa}
\newcommand{\Adv}{\mathsf{Adv}}
\newcommand{\Supp}{\mathsf{Supp}}


\newcommand{\PRG}{\mathsf{G}}
\renewcommand{\Enc}{\mathsf{Enc}}
\renewcommand{\Dec}{\mathsf{Dec}}
\renewcommand{\sk}{\mathsf{sk}}
\newcommand{\sfC}{\mathsf{C}}
\newcommand{\sfR}{\mathsf{R}}

\newcommand{\eqdef}{\stackrel{\text{\tiny def}}{=}}

\newcommand{\cF}{\mathcal{F}}

\newcommand{\angles}[1]{\langle #1 \rangle}
\newcommand{\iprod}[1]{\angles{#1}}

\newcommand{\Com}{\mathsf{Com}}


% Real vs. Ideal
\newcommand{\RealAdv}{\mathcal{A}}
\newcommand{\IdealAdv}{\mathcal{S}}
\newcommand{\RealVar}{\mathsf{Real}}
\newcommand{\IdealVar}{\mathsf{Ideal}}

% Participating parties
\newcommand{\PartyA}{P_1}
\newcommand{\PartyB}{P_2}
\newcommand{\InputA}{x_1}
\newcommand{\InputB}{x_2}


% Garbling Schemes
\newcommand{\Garble}{\mathsf{Garble}}
\newcommand{\Cir}{C}
\newcommand{\GCir}{\widetilde{C}}
\newcommand{\Lab}{\mathsf{lab}}

% Proof
\newcommand{\Sim}{\mathsf{Sim}}

% GMW

% Misc
\newcommand{\out}{\mathsf{out}}
\newcommand{\Assign}{:=}

\chapter{Secure Computation}

\section{Introduction}
Secure multiparty computation considers the problem of different parties
computing a joint function of their separate, private inputs without revealing
any extra information about these inputs than that is leaked by just the result
of the computation. This setting is well motivated, and captures many different
applications. Considering some of these applications will provide intuition
about how security should be defined for secure computation:
\begin{description}
  \item[Voting:] Electronic voting can be thought of as a multi party computation
	  between $n$ players: the voters. Their input is their choice $b \in \{0,1\}$
    (we restrict ourselves to the binary choice setting without loss of generality), and the function
    they wish to compute is the majority function.

    Now consider what happens when only one user votes: their input is trivially
    revealed as the output of the computation. What does privacy of inputs mean
    in this scenario?

  \item[Searchable Encryption:] Searchable encryption schemes allow clients
    to store their data with a server, and subsequently grant servers tokens
    to conduct specific searches. However, most schemes do not consider access
    pattern leakage. This leakage tells the server potentially valuable information
    about the underlying plaintext. How do we model all the different kinds
    information that is leaked?
\end{description}

From these examples we see that defining security is tricky, with lots of
potential edge cases to consider. We want to ensure that no party can learn
anything more from the secure computation protocol than it can from just its
input and the result of the computation. To formalize this, we adopt the
\textbf{real/ideal paradigm}.



\section{Real/Ideal Paradigm}
\paragraph{Notation.} 
Suppose there are $n$ parties, and party $P_i$ has access to some data $x_i$. They are trying to compute some function of their inputs $f(x_1, \dotsc, x_n)$. The goal is to do this securely: even if some parties are corrupted, no one should learn more than is strictly necessitated by the computation.

\paragraph{Real World.} In the real world, the $n$ parties execute a protocol $\Pi$
to compute the function $f$. This protocol can involve multiple rounds of
interaction. %Each party can additionally have some randomness.
The real world adversary $\RealAdv$ can corrupt arbitrarily many (but not all) parties.


\paragraph{Ideal World.} In the ideal world, an angel helps in the
computation of $f$:
each party sends their input to the angel and receives the output of the computation $f(x_1, \dotsc, x_n)$.
Here the ideal world adversary $\IdealAdv$ can again corrupt arbitrarily many (but not all) parties.

To model malicious adversaries, we need to modify the ideal world model as follows. 
Some parties are honest, and each honest party $P_i$ simply sends $x_i$ to the angel. The other parties are corrupted and are under control of the adversary $\IdealAdv$. The adversary chooses an input $x_i'$ for each corrupted party $P_i$ (where possibly $x_i' \neq x_i$) and that party then sends $x_i'$ to the angel. The angel computes a function $f$ of the values she receives (for example, if only party 1 is honest, then the angel computes $f(x_1, x_2', x_3', \dotsc, x_n')$) in order to obtain a tuple $(y_1, \dotsc, y_n)$. 
She then sends $y_i$ of corrupted parties to the adversary, who gets to decide whether or not honest parties will receive their response from the angel. The angel obliges. Each honest party $P_i$ then outputs $y_i$ if they receive $y_i$ from the angel and $\perp$ otherwise, and corrupted parties output whatever the adversary tells them to. 

 



\paragraph{Definition of Security.} 
A protocol $\Pi$ is secure against computationally bounded adversaries if for every PPT adversary $\RealAdv$ in the real world, there exists an PPT adversary $\IdealAdv$ in the ideal world such that for all tuples of bit strings $(x_1, \dotsc, x_n)$, we have
\[ \mathrm{Real}_{\Pi, \RealAdv}(x_1, \dotsc x_n) \stackrel{c}{\simeq} \mathrm{Ideal}_{F,\IdealAdv}(x_1, \dotsc, x_n) \]
where the left-hand side denotes the output distribution induced by $\Pi$ running with $\RealAdv$, and the right-hand side denotes the output distribution induced by running the ideal protocol $F$ with $\IdealAdv$. 
The ideal protocol is either the original one described for semi-honest adversaries, or the modified one described for malicious adversaries. 





%We require that the views of the parties
%in each of the scenarios be identical, i.e.\ that a real-world execution of the
%protocol $\Pi$ should not leak any information not leaked by the ideal-world
%execution. Hence, the parties can only learn what they can infer from their
%inputs and the output $f(\InputA, \InputB)$. More formally, assuming $\RealAdv$
%corrupts one party (say $\PartyA$, wlog), we define random variables
%$\RealVar_{\Pi, \RealAdv}(\InputA, \InputB) = \RealAdv(\InputA, r_1, \text{messages
%sent in } \Pi)$ and $\IdealVar_{F, \IdealAdv}(\InputA, \InputB) = \IdealAdv(\InputA,
%f(\InputA,\InputB))$.  These random variables represent the views of the
%adversary in each of the two settings. Our definition of security thus requires
%that
%
%\begin{equation*}
%\RealVar_{\Pi, \RealAdv}(\InputA, \InputB) \indis \IdealVar_{F, \IdealAdv}(x_1, x_2).
%\end{equation*}

\paragraph{Assumptions.} We have brushed over some details of the above setting.
Below we state these assumptions explicitly:
\begin{enumerate}
  \item \textbf{Communication channel:} We assume that the communication channel
    between the involved parties is completely insecure, i.e., it does not preserve
    the privacy of the messages. However, we assume that it is reliable, which means
    that the adversary can drop messages, but if a message is delivered, then
    the receiver knows the origin.

  \item \textbf{Corruption model:} We have different models of how and when the
    adversary can corrupt parties involved in the protocol:
    \begin{itemize}
      \item
        \emph{Static:} The adversary chooses which parties to corrupt before the
        protocol execution starts, and during the protocol, the malicious parties
        remain fixed.
      \item
        \emph{Adaptive:} The adversary can corrupt parties dynamically during
        the protocol execution, but the simulator can do the same.
      \item
        \emph{Mobile:} Parties corrupted by the adversary can be ``uncorrupted''
        at any time during the protocol execution at the adversary's discretion.
    \end{itemize}

  \item \textbf{Fairness:} The protocols we consider are not ``fair'', i.e.,
    the adversary can cause corrupted parties to abort arbitrarily. This can
    mean that one party does not get its share of the output of the computation.

  \item \textbf{Bounds on corruption:} In some scenarios, we place upper bounds
    on the number of parties that the adversary can corrupt.

  \item \textbf{Power of the adversary:} We consider primarily two types of
    adversaries:
    \begin{itemize}
      \item \emph{Semi-honest adversaries:} Corrupted parties follow the protocol
        execution $\Pi$ honestly, but attempt to learn as much information as they
        can from the protocol transcript.

      \item \emph{Malicious adversaries:} Corrupted parties can deviate arbitrarily
        from the protocol $\Pi$.
    \end{itemize}

  \item \textbf{Standalone vs.\ Multiple execution:} In some settings, protocols
    can be executed in isolation; only one instance of a particular protocol
    is ever executed at any given time. In other settings, many different protocols
    can be executed concurrently. This can compromise security.
\end{enumerate}







\section{Oblivious transfer}

\emph{Rabin's oblivious transfer} sets out to accomplish the following special task of two-party secure computation. The sender has a bit $s \in \{0,1\}$. She places the bit in a box. Then the box reveals the bit to the receiver with probability 1/2, and reveals $\perp$ to the receiver with probability 1/2. The sender cannot know whether the receiver received $s$ or $\perp$, and the receiver cannot have any information about $s$ if they receive $\perp$.

\subsection{1-out-of-2 oblivious transfer}
\emph{1-out-of-2 oblivious transfer} sets out to accomplish the following related task. The sender has two bits $s_0, s_1 \in \{0,1\}$ and the receiver has a bit $c \in \{0,1\}$. The sender places the pair $(s_0, s_1)$ into a box, and the receiver places $c$ into the same box. The box then reveals $s_c$ to the receiver, and reveals $\perp$ to the sender (in order to inform the sender that the receiver has placed his bit $c$ into the box and has been shown $s_c$). The sender cannot know which of her bits the receiver received, and the receiver cannot know anything about $s_{1-c}$.

\begin{lemma}
A system implementing 1-out-of-2 oblivious transfer can be used to implement Rabin's oblivious transfer.
\end{lemma}

\proof
The sender has a bit $s$. She randomly samples a bit $b \in \{0,1\}$ and $r \in \{0,1\}$, and the receiver randomly samples a bit $c \in \{0,1\}$. If $b = 0$, the sender defines $s_0 = s$ and $s_1 = r$, and otherwise, if $b = 1$, she defines $s_0 = r$ and $s_1 = s$. She then places the pair $(s_0, s_1)$ into the 1-out-of-2 oblivious transfer box. The receiver places his bit $c$ into the same box, and then the box reveals $s_c$ to him and $\perp$ to the sender. Notice that if $b = c$, then $s_c = s$, and otherwise $s_c = r$. Once $\perp$ is revealed to the sender, she sends $b$ to the receiver. The recieiver checks whether or not $b = c$. If $b = c$, then he knows that the bit revealed to him was $s$. Otherwise, he knows that the bit revealed to him was the nonsense bit $r$ and he regards it as $\perp$. \\

It is easy to see that this procedure satisfies the security requirements of Rabin's oblivious transfer protocol. Indeed, as we saw above, $s_c = s$ if and only if $b = c$, and since the sender knows $b$, we see that knowledge of whether or not the bit $s_c$ received by the receiver is equal to $s$ is equivalent to knowledge of $c$, and the security requirements of 1-out-of-2 oblivious transfer prevent the sender from knowing $c$. Also, if the receiver receives $r$ (or, equivalently, $\perp$), then knowledge of $s$ is knowledge of the bit that was not revealed to him by the box, which is again prevented by the security requirements of 1-out-of-2 oblivious transfer.  $\qed$

\begin{lemma}
A system implementing Rabin's oblivious transfer can be used to implement 1-out-of-2 oblivious transfer.
\end{lemma}

\proofsketch
The sender has two bits $s_0, s_1 \in \{0,1\}$ and the receiver has a single bit $c$. The sender randomly samples $3n$ random bits $x_1, \dotsc, x_{3n} \in \{0,1\}$. Each bit is placed into its own a Rabin oblivious transfer box. The $i$th box then reveals either $x_i$ or else $\perp$ to the receiver. Let 
\[ S := \{i \in \{1, \dotsc, 3n\} : \text{the receiver knows } x_i\}. \]
The receiver picks two sets $I_0, I_1 \subseteq \{1, \dotsc, 3n\}$ such that $\# I_0 = \# I_1 = n$, $I_c \subseteq S$ and $I_{1-c} \subseteq \{1, \dotsc, 3n\} \setminus S$. This is possible except with probability negligible in $n$. He then sends the pair $(I_0, I_1)$ to the sender. The sender then computes $t_j= \left(\bigoplus_{i \in I_j}x_i \right) \oplus s_j$ for both $j \in \{0,1\}$ and sends $(t_0, t_1)$ to the receiver. \\

Notice that the receiver can uncover $s_c$ from $t_c$ since he knows $x_i$ for all $i \in I_c$, but cannot uncover $s_{1-c}$. One can show that the security requirement of Rabin's oblivious transfer implies that this system satisfies the security requirement necessary for 1-out-of-2 oblivious transfer. $\qed$ \\

We will see below that length-preserving one-way trapdoor permutations can be used to realize 1-out-of-2 oblivious transfer. 

\begin{theorem}
The following protocol realizes 1-out-of-2 oblivious transfer in the presence of computationally bounded and semi-honest adversaries. 
\begin{enumerate}
\item The sender, who has two bits $s_0$ and $s_1$, samples a random length-preserving one-way trapdoor permutation $(f, f^{-1})$ and sends $f$ to the receiver.  Let $b(\cdot)$ be a hard-core bit for $f$.
\item The receiver, who has a bit $c$, randomly samples an $n$-bit string $x_c \in \{0,1\}^n$ and computes $y_c = f(x_c)$. He then samples another random $n$-bit string $y_{1-c} \in \{0,1\}^n$, and then sends $(y_0, y_1)$ to the sender.
\item The sender computes $x_0 := f^{-1}(y_0)$ and $x_1 := f^{-1}(y_1)$. She computes $b_0 := b(x_0) \oplus s_0$ and $b_1 := b(x_1) \oplus s_1$, and then sends the pair $(b_0, b_1)$ to the receiver.
\item The receiver knows $c$ and $x_c$, and can therefore compute $s_c = b_c \oplus b(x_c)$. 
\end{enumerate}
\end{theorem}
\proof
Correctness is clear from the protocol.	
For security, from the sender side, since $f$ is a length-preserving permutation, $(y_0, y_1)$ is statistically indistinguishable from two random strings, hence she can't learn anything about $c$.
From the receiver side, guessing $s_{1-c}$ correctly is equivalent to guessing the hard-core bit for $y_{1-c}$.
\qed


\subsection{1-out-of-4 oblivious transfer}
  We describe how to implement a 1-out-of-4 OT using 1-out-of-2 OT:\@
  \begin{enumerate}
    \item
      The sender, $\PartyA$ samples 5 random values $S_i \gets \bits, i \in \set{1,\dotsc, 5}$.
    \item
      $\PartyA$ computes
      \begin{align*}
        \alpha_0 &= S_0 \xor S_2 \xor m_0\\
        \alpha_1 &= S_0 \xor S_3 \xor m_1\\
        \alpha_2 &= S_1 \xor S_4 \xor m_2\\
        \alpha_3 &= S_1 \xor S_5 \xor m_3
      \end{align*}
      It sends these values to $\PartyB$.
    \item
      The parties engage in 3 1-out-of-2 Oblivious Transfer protocols for the following
      messages: $(S_0, S_1)$, $(S_2, S_3)$, $(S_4, S_5)$. THe receiver's input for
      the first OT is the first choice bit, and for the second and third ones is
      the second choice bit.
    \item
      The receiver can only decrypt one ciphertext.
  \end{enumerate}
