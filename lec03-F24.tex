

\section{Levin's One-Way Function}

In this section, we discuss Levin's one-way function, which is an explicit construction of a one-way function that is secure as long as a one-way function exists.
This is interesting because unlike a typical cryptographic primitive that relies on a specific hardness assumption (which may or may not hold in the future), Levin's one-way function is future-proof in the sense that it will be secure as long as atleast one hardness assumption holds (which we may or may not discover).

The high-level intuition behind Levin's construction is as follows: since we assume one-way functions exist, there exists a uniform machine $\tilde M$ such that $|\tilde M|$ is a constant and $\tilde M(x)$ is hard to invert for a random input $x$.
Now, consider a function $h$ that parses the first $\log(n)$ bits of its $n$-bit input as the code of a machine $M$ and the remaining bits as the input to $M$.
For a large enough $n$ that is exponential in $|\tilde{M}|$, note that we will hit the code of $\tilde{M}$ with noticeable probability in $n$, and for those instances, $h$ will be hard to invert.
% that if you parse the input to the function as the code of a uniform machine $M$ followed by the input to the machine, for large enough input length $n$ that is exponential in $|M|$, you'll get the code of a one-way function with some noticeable probability.
It is easy to see that this gives us a weak one-way function which has a noticeable probability of being hard to invert, and we can amplify the hardness of this weak one-way function to get an explicit construction of a one-way function.

\begin{theorem}\label{thm:levin}
If there exists a one-way function, then there exists an explicit function $f$ that is one-way  (constructively).
\end{theorem}

Before we look at the construction and the proof in detail, we first prove a lemma that will be useful in the proof.
In particular, we need a bound on the running time of the one-way function $\tilde M$ so that we can upper bound the execution time of $h$, since there could be inputs to $g$ that do not terminate in polynomial time.
To this end, we prove the following lemma which shows that if a one-way function exists, then there is also a one-way function that runs in time $n^2$, and thus, we can bound $h$ to $n^2$ steps.

\begin{lemma}\label{lem:n2owf}
If there exists a one-way function computable in time $n^c$ for a constant $c$, then there exists a one-way function computable in time $n^2$.
\end{lemma}
\proof
Let $f: \binset{n} \rightarrow \binset{n}$ be a one-way function computable in time $n^c$.
Construct $g: \binset{n+n^c} \rightarrow \binset{n+n^c}$ as follows:
$$g(x,y) = f(x) || y$$
where $x \in \binset{n}, y \in \binset{n^c}$.
$g(x,y)$ takes time $2n^c$, which is linear in the input length.

We next show that $g(\cdot)$ is one-way.
Assume for the purpose of contradiction that there exists an adversary $\mathcal{A}$ such that $\mu_{\mathcal{A},g}(n+n^c) = \Pr_{(x,y) \stackrel{\$}{\leftarrow} \binset{n+n^c}}[ \mathcal{A}(1^{n+n^c}, g(x,y)) \in g^{-1}(g(x,y))]$ is non-negligible. Then we use $\mathcal{A}$ to construct $\mathcal{B}$ such that $\mu_{\mathcal{B},f}(n) = \Pr_{x \stackrel{\$}{\leftarrow} \binset{n}}[ \mathcal{B}(1^n, f(x)) \in f^{-1}(f(x))]$ is also non-negligible.

$\mathcal{B}$ on input $z \in\{0,1\}^n$, samples $y \stackrel{\$}{\leftarrow} \binset{n^c}$, and outputs the $n$ higher-order bits of  $\mathcal{A}(1^{n+n^c}, z||y)$. Then we have
\begin{align*}
\mu_{\mathcal{B},g}(n) =& \Pr_{x \stackrel{\$}{\leftarrow} \binset{n}, y \stackrel{\$}{\leftarrow} \binset{n^c}}\left[\mathcal{A}(1^{n+n^c}, f(x) || y) \in f^{-1}(f(x)) || \binset{n^c}\right]\\
\geq&\Pr_{x,y}\left[\mathcal{A}(1^{n+n^c}, g(x,y)) \in f^{-1}(f(x)) || y\right]\\
=& \Pr_{x,y}\left[\mathcal{A}(1^{n+n^c}, g(x,y)) \in g^{-1}(g(x,y))\right]
\end{align*}
is non-negligible.
\qed

\bigskip
Now, we provide the explicit construction of $h$ and prove that it is a weak one-way function.
Since $h$ is an (explicit) weak one-way function, we can construct an (explicit) one-way function from $h$ as we discussed in Section~\ref{sec:owf:amplify}, and this would prove Theorem~\ref{thm:levin}.
\proof[Proof of Theorem~\ref{thm:levin}]
$h: \binset{n} \rightarrow \binset{n}$ is defined as follows:
$$
h(M,x) = \left\{
\begin{array}{ll}
M || M(x) & \text{if $M(x)$ takes no more than $|x|^2$ steps} \\
M || 0 & \text{otherwise}
\end{array}
\right.
$$
where $|M| = \log n, |x| = n - \log n$ (interpreting $M$ as the code of a machine  and $x$ as its input).

It remains to show that if one-way functions exist, then $h$ is a weak one-way function, with $\alpha_h(n) = \frac{1}{n^2}$.
Assume for the purpose of contradiction that there exists an adversary $\mathcal{A}$ such that $\mu_{\mathcal{A},h}(n) = \Pr_{(M,x) \stackrel{\$}{\leftarrow} \binset{n}}[ \mathcal{A}(1^{n}, h(M,x)) \in h^{-1}(h(M,x))]\geq 1-\frac{1}{n^2}$ for all sufficiently large $n$.
By the existence of one-way functions and Lemma~\ref{lem:n2owf}, there exists a one-way function $\tilde M$ that can be computed in time $n^2$. Let $\tilde M$ be the uniform machine that computes this one-way function.
We will consider values $n$ such that $n > 2^{|\tilde M|}$. In other words for these choices of $n$, $\tilde M$ can be described using $\log n$ bits.
We construct $\mathcal{B}$ to invert $\tilde M$: on input $y$ outputs the $(n-\log n)$ lower-order bits of $\mathcal{A}(1^n, \tilde M||y)$. Then
\begin{align*}
\mu_{\mathcal{B},\tilde M}(n-\log n) =& \Pr_{x \stackrel{\$}{\leftarrow} \binset{n-\log n}}\left[\mathcal{A}(1^{n}, \tilde M || \tilde M(x)) \in \binset{\log n} || \tilde M^{-1}(\tilde M((x))\right]\\
\geq& \Pr_{x \stackrel{\$}{\leftarrow} \binset{n-\log n}}\left[\mathcal{A}(1^{n}, \tilde M || \tilde M(x)) \in \tilde{M} || \tilde M^{-1}(\tilde M((x))\right].
\end{align*}
Observe that for sufficiently large $n$ it holds that
\begin{align*}
1-\frac{1}{n^2} \leq& \mu_{\mathcal{A},h}(n)\\
=& \Pr_{(M,x) \stackrel{\$}{\leftarrow} \binset{n}}\left[ \mathcal{A}(1^{n}, h(M,x)) \in h^{-1}(h(M,x))\right]\\
\leq& \Pr_{M }[M = \tilde M] \cdot \Pr_{x }\left[ \mathcal{A}(1^{n}, \tilde M||\tilde M(x)) \in  \tilde{M} || \tilde M^{-1}(\tilde M((x))\right] + \Pr_{M }[M \neq \tilde M]  \\
\leq&  \frac{1}{n} \cdot \mu_{\mathcal{B},\tilde M}(n-\log n) +\frac{n-1}{n}.
\end{align*}
Hence $\mu_{\mathcal{B},\tilde M}(n-\log n) \geq \frac{n-1}{n}$  for sufficiently large $n$ which is a contradiction.
\qed

\section{Hardness Concentrate Bit}
We start by asking the following question: Is it possible to concentrate the strength of a one-way function into one bit? In particular, given a one-way function $f$, does there exist one bit that can be computed efficiently from the input $x$, but is hard to compute given $f(x)$?
\begin{definition}[Hard Concentrate Bit]
Let $f:\binset{n} \rightarrow \binset{n}$ be a one-way function.
$B:\{0,1\}^n \rightarrow \{0,1\}$ is a hard concentrate bit of $f$ if:
\begin{itemize}
\item[-] $B$ is computable by a polynomial time machine, and
\item[-] $\forall$ non-uniform PPT adversaries $\mathcal{A}$ we have that
	$$\Pr_{x\stackrel{\$}{\leftarrow} \binset{n}}[\mathcal{A}(1^n, f(x)) = B(x)] \leq \frac{1}{2} + \mathsf{negl}(n).$$
\end{itemize}
\end{definition}

\noindent\textbf{A simple example.}
Let $f$ be a one-way function. Consider the one-way function $g(b, x) = 0 || f(x)$ and a hard concentrate bit $B(b, x) = b$.
Intuitively, the value $g(b, x)$ does not reveal any information about the first bit $b$, thus no information about the value $B(b, x)$ can be ascertained. Hence $\mathcal{A}$ cannot predict the first bit with a non-negligible advantage than a random guess. However, we are more interested in the case where the hard concentrate bit is hidden because of computational hardness and not information theoretic hardness.

\bigskip
\begin{remark}
Given a one-way function $f$, we can construct another one-way function $g$ with a hard concentrate bit. However, we may not be able to find a hard concentrate bit for $f$. In fact, it is an open question whether a hard concentrate bit exists for every one-way function.
\end{remark}

\bigskip
Intuitively, if a function $f$ is one-way, it seems that there should be a particular bit in the input $x$ that is hard to compute given $f(x)$. However, we show that is not true:
\begin{claim}
If $f:\binset{n}\rightarrow \binset{n}$ is a one-way function, then there exists a one-way function $g:\binset{n+\log n}\rightarrow\binset{n+\log n}$ such that $\forall i \in [1, n+\log n]$, $B_i(x) = x_i$ is not a hard concentrate bit, where $x_i$ is the $i^\text{th}$ bit of $x$.
\end{claim}
\proof
Define $g:\{0,1\}^{n+\log(n)} \rightarrow \{0,1\}^{n+\log(n)}$ as follows.
$$g(x,y) = f(x_{\bar y}) || x_y || y,$$
where $|x| = n, |y| = \log n$, $x_{\bar y}$ is all bits of $x$ except the $y^\text{th}$ bit, and $x_y$ is the $y^\text{th}$ bit of $x$.

First, one can show that $g$ is still a one-way function. (We leave this as an exercise!)
Next, we show that $B_i$ is not a hard concentrate bit for $\forall i \in [1, n]$ (clearly $B_i$ is not a hard concentrate bit for $i \in [n+1, n+\log n]$).
Construct an adversary $\mathcal{A}_i(1^{n+\log n}, f(x_{\bar y}) || x_y || y)$ that ``breaks'' $B_i$:
\begin{itemize}
\item[-] If $y \not= i$ then output a random bit;
\item[-] Otherwise output $x_y$.
\end{itemize}
\begin{align*}
& \Pr_{x, y}[\mathcal{A}(1^{n+\log n}, g(x,y)) = B_i(x)]\\
=& \Pr_{x, y}[\mathcal{A}(1^{n+\log n}, f(x_{\bar y}) || x_y || y) = x_i]\\
=& \frac{n-1}{n} \cdot \frac{1}{2} + \frac{1}{n} \cdot 1 = \frac{1}{2} + \frac{1}{2n}.
\end{align*}
Hence $\mathcal{A}_i$ can guess the output of $B_i$ with greater than $\frac{1}{2} + \mathsf{negl}(n)$ probability.
\qed


\iffalse
\paragraph{Application: Coin tossing over the phone.} We next describe an application of hard concentrate bits to coin tossing.
Consider two parties trying to perform a coin tossing over the phone. In this setting the first party needs to declare its choice as the second one flips the coin. However, how can the first party trust the win/loss response from the second party?  In particular, if the first party calls out ``head'' and then the second party can just lie that it was ``tails.'' We can use hard concentrate bit of a (one-to-one) one-way function to  enable this applications.

Let $f$ be a (one-to-one) one-way function and $B$ be a hard concentrate bit for $f$. Consider the following protocol:
\begin{itemize}
\item[-] Party $P_1$ samples $x$ from $\{0,1\}^n$ uniformly at random and sends $y$, where $y = f(x)$, to party $P_2$.
\item[-] $P_2$ sends back a random bit $b$ sampled from $\{0,1\}$.
\item[-] $P_1$ sends back $(x, B(x))$ to $P_2$. $P_2$ aborts if $f(x) \neq y$.
\item[-]  Both parties output $B(x)\oplus b$.
\end{itemize}
Note that $P_2$ cannot guess $B(x)$ with a non-negligible advantage than $1/2$ as he sends back his $b$.
On the other hand, $P_1$ cannot flip the value $B(x)$ once it has sent $f(x)$ to $P_2$ because $f$ is one-to-one.
\fi

\subsection{Hard Concentrate Bit of any One-Way Permutation}
We now show that a slight modification of every one-way function has a hard concentrate bit. More formally,
\begin{theorem}\label{thm:hard-concentrate-bit}
Let  $f:\binset{n} \rightarrow \binset{n}$ be a one-way function.
Define a function $g:\binset{2n} \rightarrow \binset{2n}$ as follows:
$$g(x,r) = f(x) || r,$$
where $|x| = |r| =n$. Then we have that $g$ is one-way and that it has a hard concentrate bit, namely $B(x, r) = \sum_{i=1}^n x_i r_i\mod 2$.
\end{theorem}
\begin{remark}
If $f$ is a (one-to-one) one-way function, then $g$ is also a (one-to-one) one-way function with hard concentrate bit $B(\cdot)$.
\end{remark}
\proof
We leave it as an exercise to show that $g$ is a one-way function and below we will prove that the function $B(\cdot)$ describe a hard concentrate bit of $g$.
More specifically, we need to show that if there exists a non-uniform PPT  $\ma$ s.t. $\Pr_{x,r}[\ma(1^{2n},g(x,r)) = B(x,r)] \ge \frac{1}{2} + \epsilon(n)$, where $\epsilon$ is non-negligible, then there exists a non-uniform PPT $\mathcal{B}$ such that $\Pr_{x,r}[\mathcal{B}(1^{2n}, g(x,r)) \in g^{-1}(g(x,r))]$ is non-negligible.
Below we use $E$ to denote the event that $\ma(1^{2n},g(x,r)) = B(x,r)$.
We will present our proof in three steps, where each step progressively increases in complexity:
(1) the super simple case where we restrict to $\ma$ such that $\Pr_{x,r}[E] = 1$,
(2) the simple case where we restrict to $\ma$ such that $\Pr_{x,r}[E] \geq \frac34 + \epsilon(n)$,
and finally (3) the general case where $\Pr_{x,r}[E] \geq \frac12 + \epsilon(n)$.

\medskip
\noindent\textbf{\underline{Super simple case.}}
Suppose $\ma$ guesses $B(\cdot)$ with perfect accuracy:
$$\Pr_{x,r}[E] =1.$$
We now construct $\mathcal{B}$ that inverts $g$ with perfect accuracy.
Let $e^i$ denote the one-hot $n$-bit string $0\cdots 0 1 0 \cdots0$, where only the $i$-th bit is $1$, the rest are all $0$.
$\mathcal{B}$ gets $f(x)||r$ as input, and its algorithm is described in Figure~\ref{alg:super-simple-case}.

\begin{marginfigure}
\begin{algorithmic}
\For {$i=1$ \textbf{to} $n$}
    \State $x'_i \gets \ma(1^{2n}, f(x)||e^i)$
\EndFor
\State \Return $x'_1\cdots x'_n || r$
\end{algorithmic}
\caption{Super-Simple Case $\mathcal{B}$} \label{alg:super-simple-case}
\end{marginfigure}
Observe that $B(x,e^i) = \sum_{j=1}^n x_je^i_j = x_i$. Therefore, the probability that $\mathcal{B}$ inverts a single bit successfully is,
$$\Pr_{x}\left[\ma(1^{2n}, f(x)||e^i)=x_i\right] =  \Pr_{x}\left[\ma(1^{2n}, f(x)||e^i)=B(x,e^i)\right] = 1.$$
Hence $\Pr_{x,r}[\mathcal{B}(1^{2n}, g(x,r)) = (x,r)] = 1$.


\bigskip
\noindent\textbf{\underline{Simple case.}}
Next moving on to the following more demanding case.
$$\Pr_{x,r}[E] \geq \frac{3}{4} + \epsilon(n),$$ where $\epsilon(\cdot)$ is non-negligible.
We describe $\mathcal{B}$'s algorithm for inverting $g$ in Figure~\ref{alg:simple-case}.
Here we can no longer use the super simple case algorithm because we no longer know if $\ma$ outputs the correct bit on input $f(x) \| e^i$.
Instead, we introduce randomness to $\ma$'s input expecting that it should be able to guess the right bit on majority of those inputs since it has a high probability of guessing $B(\cdot)$ in general.
We now also need to make two calls to $\ma$ to isolate the $i$-th bit of $x$.
% On input $f(x)||r$, $\mathcal{B}$ proceeds as follows:
Note that an iteration of $\mathcal{B}$ outputs the right bit if calls to $\ma$ output the correct bit because $B(x,s) \oplus B(x, s\oplus e^i) = x_i$:
\begin{marginfigure}
\begin{algorithmic}
\For {$i = 1$ \textbf{to} $n$}
	\For {$t = 1$ \textbf{to} $T = \frac{n}{2\epsilon(n)^2}$}
		\State $s \stackrel{\$}{\leftarrow} \binset{n}$
    	% \State $x_i^t \leftarrow \ma(f(x)|| s) \oplus \ma(f(x) || s+e^i)$
		\State $\begin{array}{l@{}l}
		x_i^t \leftarrow& \ma(f(x)|| s) \\
		& \oplus~ \ma(f(x) || (s\oplus e^i))
		\end{array}$
	\EndFor
	\State $x'_i \gets $ the majority of $\{x_i^1, \cdots, x_i^T\}$
\EndFor
\State \Return $x'_1\cdots x'_n||R$
\end{algorithmic}
\caption{Simple Case $\mathcal{B}$} \label{alg:simple-case}
\end{marginfigure}
\begin{align*}
B(x,s) \oplus B(x, s\oplus e^i) =& \sum_j x_j s_j \oplus \sum_j x_j (s_j \oplus e^i_j)\\
=& \sum_{j \not= i} (x_j s_j \oplus x_j s_j) \oplus x_i s_i \oplus x_i (s_i \oplus 1)\\
=& ~x_i
\end{align*}
The key technical challenge in proving that $\mathcal{B}$ inverts $g$ with non-negligible probability arises from the fact that the calls to $\ma$ made during one iteration of $\mathcal{B}$ are not independent.
In particular, all calls to $\ma$ share the same $x$ and the calls $\ma(f(x)|| s)$ and $\ma(f(x) || (s \oplus e^i))$ use correlated randomness as well.

We solve the first issue by showing that there exists a large set of $x$ values for which $\ma$ still works with large probability.
The latter issue of lack of independence between $\ma(f(x)|| s)$ and $\ma(f(x) || (s \oplus e^i))$ can be solved using a union bound since the success probability of the adversary $\ma$ is high enough.

Formally, define the set $G$ of ``good'' $x$'s, for which it is easy for $\ma$ to predict the right bit:
$$G := \left\{x \left| \Pr_r \left[ E \right]\geq \frac{3}{4} + \frac{\epsilon(n)}{2} \right. \right\}.$$
Now we prove that $G$ is not a small set. More formally, we claim that:
$$\Pr_{x \stackrel{\$}{\leftarrow} \binset{n}}[x \in G] \geq \frac{\epsilon(n)}{2}.$$
Assume that $\Pr_{x \stackrel{\$}{\leftarrow} \binset{n}}[x \in G] < \frac{\epsilon(n)}{2}$. Then we have the following contradiction:
\begin{align*}
\frac{3}{4} + \epsilon(n) \leq&~ \Pr_{x,r}[E]\\
=&~ \Pr_x [x \in G] \Pr_{r}[E| x\in G] + \Pr_x [x \notin G] \Pr_{r}[E | x\notin G]\\
< &~ \frac{\epsilon(n)}{2} \cdot 1 + 1\cdot \left(\frac{3}{4}+\frac{\epsilon(n)}{2}\right)  = \frac{3}{4} + \epsilon(n).
\end{align*}
Now consider a single iteration for a fixed $x \in G$:
\begin{align*}
&\Pr_{s} \left[ \ma(f(x), s) \oplus \ma(f(x), s\oplus e^i) = x_i \right]\\
&=~ \Pr_{s} \left[ \text{Both $\ma$'s are correct} \right] + \Pr_{s} \left[ \text{Both $\ma$'s are wrong} \right]\\
&\geq~ \Pr_{s} \left[ \text{Both $\ma$'s are correct} \right] = 1- \Pr_{s} \left[ \text{Either $\ma$ is wrong} \right]\\
&\geq~ 1- 2\cdot\Pr_{s} \left[ \text{$\ma$ is wrong} \right]\\
&\geq~ 1-2\left(\frac{1}{4} - \frac{\epsilon(n)}{2} \right) 
= \frac{1}{2} + \epsilon(n).
\end{align*}
Let $Y_i^t$ be the indicator random variable that $x_i^t = x_i$ (namely, $Y_i^t=1$ with probability $\Pr[x_i^t = x_i]$ and $Y_i^t=0$ otherwise).
Note that $Y_i^1, \cdots, Y_i^T$ are independent and identical random variables, and for all $t \in \{1,\ldots, T\}$, we have $\Pr[Y_i^t=1] = \Pr[x_i^t = x_i] \geq \frac{1}{2} + \epsilon(n)$.
Next we argue that majority of $x_i^t$ coincide with $x_i$ with high probability.
\begin{align*}
\Pr[x'_i \neq x_i]
=&~ \Pr\left[\sum_{t=1}^T Y_i^t \leq \frac{T}{2} \right]\\
=&~ \Pr\left[\sum_{t=1}^T  Y_i^t- \left(\frac{1}{2} + \epsilon(n) \right)T \leq \frac{T}{2} - \left(\frac{1}{2} + \epsilon(n) \right)T \right]\\
\leq&~ \Pr\left[ \left| \sum_{t=1}^T  Y_i^t- \left(\frac{1}{2} + \epsilon(n) \right)T \right| \geq \epsilon(n)T \right]\\
& \text{Let $X_1,\cdots,X_m$ be i.i.d. random variables taking values 0 or 1. Let $\Pr[X_i=1] = p$.}\\
& \text{By Chebyshev's Inequality, $\Pr\left[ \left| \sum X_i - pm \right| \geq \delta m \right] \leq \frac{1}{4\delta^2 m}$.}\\
\leq&~ \frac{1}{4\epsilon(n)^2T} = \frac{1}{2n}.
\end{align*}
Then, completing the argument, we have
\begin{align*}
&\Pr_{x,r}[\mathcal{B}(1^{2n}, g(x,r)) = (x,r)]\\
&\geq~ \Pr_x [x \in G] \Pr[x'_1 = x_1, \cdots x'_n = x_n | x \in G]\\
&\geq~ \frac{\epsilon(n)}{2} \cdot \left(1- \sum_{i=1}^n\Pr[x'_i \neq x_i | x \in G]\right)\\
&\geq~ \frac{\epsilon(n)}{2} \cdot \left(1- n \cdot\frac{1}{2n} \right) =  \frac{\epsilon(n)}{4}.
\end{align*}



\bigskip
\noindent\textbf{\underline{Real Case.}} Now, we describe the final case where $\Pr_{x,r}[E] \geq \frac{1}{2} + \epsilon(n)$ and $\epsilon(\cdot)$ is a non-negligible function.
The key technical challenge in this case is that we cannot make two related calls to $\ma$ as was done in the simple case above since we can't argue that both calls to $\ma$ will be correct with high enough probability.
However, just using one call to $\ma$ seems insufficient.
The key idea is to just guess one of those values.
Very surprisingly, this idea along with careful analysis magically works out.
Just like the previous two cases, we start by describing the algorithm $\mathcal{B}$ in Figure~\ref{alg:real-case}.


\begin{marginfigure}
\begin{algorithmic}
\State $T = \frac{2n}{\epsilon(n)^2}$
	\For {$\ell = 1$ \textbf{to} $\log T$}
		\State $s_\ell \stackrel{\$}{\leftarrow} \binset{n}$
		\State $b_\ell \stackrel{\$}{\leftarrow} \{0,1\}$
	\EndFor
\For {$i = 1$ \textbf{to} $n$}
	\ForAll {$L \subseteq \{1,2,\cdots, \log T\}$}
		\State $S_L :=\bigoplus_{j \in L} s_j$
		\State $B_L := \bigoplus_{j \in L} b_j$
		\State $x_i^L \leftarrow B_L \oplus \ma(f(x) || S_L\oplus e^i)$
	\EndFor
	\State $x'_i \gets $ majority of $\{x_i^\emptyset, \cdots, x_i^{[\log T]}\}$
\EndFor
\State \Return $x'_1\cdots x'_n||R$
\end{algorithmic}
\caption{Real Case $\mathcal{B}$} \label{alg:real-case}
\end{marginfigure}

\medskip
In the beginning of the algorithm, $\mathcal{B}$ samples $\log T$ random strings $\{ s_\ell \}_{\ell}$ and bits $\{ b_\ell \}_{\ell}$.
Since there are only $\log T$ values, with probability $\frac{1}{T}$ (which is polynomial in $n$) all the $b_{\ell}$'s are correct, i.e., $b_\ell = B(x, s_\ell)$. In the rest of this proof, we denote this event as $F$.
Now note that if $F$ happens, then $B_L$ as defined in the algorithm is also equal to $B(x, S_L)$ (we denote the $k^{\text{th}}$-bit of $s$ with $(s)_k$): 

\begin{align*}
B(x,S_L) &= \sum_{k=1}^n x_k (\bigoplus_{j \in L} s_j)_k \\ 
         &=  \sum_{k=1}^n x_k \sum_{j \in L} \left(s_j\right)_k \\
         &= \sum_{j \in L} \sum_{k=1}^n x_k (s_j)_k \\
         &= \sum_{j \in L} B(x,s_j) \\
         &= \sum_{j\in L} b_j  \\
         &= B_L
\end{align*}

Thus, with probability $\frac{1}{T}$, we have all the right guesses for one of the invocations, and we just need to bound the probability that $\ma(f(x) || S_L \oplus e^i) = B(x, S_L \oplus e^i)$.
However there is a subtle issue. Now the events $Y_i^\emptyset, \cdots, Y_i^{[\log T]}$ are no longer independent.
Nevertheless, we can still show that they are pairwise independent, and the Chebyshev's Inequality still holds. Now we give the formal proof.

Just as in the simple case, we define the set $G$ as
$$G := \left\{x \left| \Pr_r \left[ E \right]\geq \frac{1}{2} + \frac{\epsilon(n)}{2} \right. \right\},$$
and with an identical argument we obtain that:

$$\Pr_{x \stackrel{\$}{\leftarrow} \binset{n}}[x \in G] \geq \frac{\epsilon(n)}{2}$$


\marginnote[-2cm]{%
\noindent\textbf{Pairwise Independence and Chebyshev's Inequality.} For the sake of completeness, we prove the Chebyshev's Inequality here.
\begin{definition}[Pairwise Independence]
A collection of random variables $\{X_1,\cdots,X_m\}$ is said to be \emph{pairwise independent} if for every pair of random variables $(X_i, X_j), i \neq j$  and every pair of values $(v_i,v_j)$, it holds that
\[\Pr[X_i = v_i, X_j = v_j] = \Pr[X_i = v_i]\Pr[X_j = v_j]\]
\end{definition}

\begin{theorem}[Chebyshev's Inequality]\label{thm:Chebyshev}
Let $X_1,\hdots,X_m$ be pairwise independent and identically distributed binary random variables. In particular, for every $i \in [m]$, $\Pr[X_i = 1] = p$ for some $p\in [0,1]$ and $\Pr[X_i=0]=1-p$. Then it holds that
$$\Pr\left[\left|\sum_{i=1}^m X_i - pm\right| \geq \delta m\right] \leq \frac{1}{4\delta^2m}.$$
\end{theorem}

\proof
Let $Y = \sum_i X_i$. Then
\begin{align*}
&\Pr\left[\left|\sum_{i=1}^m X_i - pm\right| > \delta m\right] \\
&=~ \Pr\left[\left(\sum_{i=1}^m X_i - pm\right)^2> \delta^2 m^2\right]\\
&\leq~ \frac{\mathbb{E}\left[\left|Y - pm\right|^2\right]}{\delta^2m^2} = \frac{\text{Var}(Y)}{\delta^2m^2} \\
\end{align*}
Observe that
\begin{align*}
\text{Var}(Y) &= \mathbb{E}\left[Y^2\right] - \left(\mathbb{E}[Y]\right)^2\\
&= \sum_{i=1}^m \sum_{j=1}^m \left( \mathbb{E}\left[X_iX_j\right] - \mathbb{E}\left[X_i\right] \mathbb{E}\left[X_j\right]\right)\\
& \text{By pairwise independence, for $i \neq j$,} \\  
& \text{$\mathbb{E}\left[X_i X_j\right] = \mathbb{E}\left[X_i\right] \mathbb{E}\left[X_j\right]$.}\\
&= \sum_{i=1}^m \mathbb{E}\left[X_i^2\right] - \mathbb{E}\left[X_i\right]^2\\
&= mp(1-p).
\end{align*}
Hence
$$\Pr\left[\left|\sum_{i=1}^m X_i - pm\right| \geq\delta m\right] \leq \frac{mp(1-p)}{\delta^2m^2} \leq \frac{1}{\delta^2m}.$$
\qed
}

% Correctness of $\mathcal{B}$ follows from the fact in case $b_\ell = B(x,s_\ell)$ for every $\ell \in [\log T]$ then $\forall L \subseteq [\log T]$, it holds that
Next, given $\{ b_\ell = B(x,s_\ell) \}_{\ell \in [\log T]}$ and $x\in G$, we have:
\begin{align*}
&\Pr_{r} \left[  B_L \oplus \ma(f(x) || S_L \oplus e^i) = x_i \right] \\
&=~ \Pr_{r} \left[ B(x,S_L) \oplus \ma(f(x) || S_L \oplus e^i) = x_i \right]\\
&=~ \Pr_{r} \left[ \ma(f(x) || S_L \oplus e^i) =  B(x,S_L \oplus e^i) \right]\\
&\geq~ \frac{1}{2} + \frac{\epsilon(n)}{2}
\end{align*}
For the same $\{ b_\ell \}_\ell$ and $x\in G$, let $Y_i^L$ be the indicator random variable that $x_i^L = x_i$.
Notice that $Y_i^\emptyset, \cdots, Y_i^{[\log T]}$ are pairwise independent and $\Pr[Y_i^L=1] = \Pr[x_i^L = x_i] \geq \frac{1}{2} + \frac{\epsilon(n)}{2}$.
\begin{align*}
\Pr[x'_i \neq x_i] =& \Pr\left[\sum_{L \subseteq [\log T]} Y_i^L \leq \frac{T}{2} \right]\\
=& \Pr\left[\sum_{L \subseteq [\log T]} Y_i^L - \left(\frac{1}{2} +  \frac{\epsilon(n)}{2} \right)T \leq \frac{T}{2} - \left(\frac{1}{2} +  \frac{\epsilon(n)}{2} \right)T \right]\\
\leq& \Pr\left[ \left| \sum_{L \subseteq [\log T]} Y_i^L - \left(\frac{1}{2} +  \frac{\epsilon(n)}{2} \right)T \right| \geq \frac{\epsilon(n)}{2} T \right]\\
& \text{(By Theorem~\ref{thm:Chebyshev})}\\
\leq& \frac{1}{4\left( \frac{\epsilon(n)}{2}\right)^2T} = \frac{1}{2n}.
\end{align*}
Completing the proof, we have that:
\begin{align*}
& \Pr_{x,r}[\mathcal{B}(1^{2n}, g(x,r)) = (x,r)]\\
&\geq~  \Pr_{\{ b_\ell, s_\ell \}_\ell}\left[ F \right] \cdot  \Pr_x [x \in G] \cdot \Pr[x'_1 = x_1, \cdots x'_n = x_n ~|~ x \in G ~\wedge~ F]\\
&\geq~ \frac{1}{T} \cdot \frac{\epsilon(n)}{2} \cdot \left(1- \sum_{i=1}^n\Pr[x'_i \neq x_i ~|~  x \in G ~\wedge~ F]\right)\\
&\geq~ \frac{\epsilon(n)^2}{2n} \cdot \frac{\epsilon(n)}{2} \cdot \left(1- n \cdot\frac{1}{2n} \right) =  \frac{\epsilon(n)^3}{8n}
\end{align*}
\qed



\newpage
\section*{Exercises}
\begin{exercise}
\label{ex:product} If $\mu(\cdot)$ and $\nu(\cdot)$ are negligible functions then show that $\mu(\cdot) \cdot \nu(\cdot)$ is a negligible function.
\end{exercise}

\begin{exercise}
\label{ex:product} If $\mu(\cdot)$ is a negligible function and $f(\cdot)$ is a function polynomial in its input then show that $\mu(f(\cdot))$\footnote{Assume that $\mu$ and $f$ are such that $\mu(f(\cdot))$ takes inputs from $\mathbb{Z}^+$ and outputs values in $[0,1]$.} are negligible functions.
\end{exercise}

\begin{exercise}\label{ex:PNP} Prove that the existence of one-way functions implies $P \neq NP$.
\end{exercise}

\begin{exercise}
Prove that there is no one-way function $f:\{0,1\}^n\to \{0,1\}^{\lfloor \log_2 n\rfloor}$.
\end{exercise}


\begin{exercise} Let $f:\{0,1\}^n\to \{0,1\}^{n}$ be any one-way function then is $f'(x) \stackrel{def}{=} f(x)\oplus x$ necessarily one-way?
\end{exercise}

\begin{exercise}
Prove or disprove: If $f: \{0,1\}^n\rightarrow \{0,1\}^n$ is a one-way function, then $g: \{0,1\}^n\rightarrow \{0,1\}^{n-\log n}$ is a one-way function, where $g(x)$ outputs the $n-\log n$ higher order bits of $f(x)$.
\end{exercise}

\begin{exercise}
Explain why the proof of Theorem~\ref{theorem:weakstrongOWF} fails if the attacker $\mathcal{A}$ in Figure~\ref{fig:adv:weak} sets $i = 1$ and not $i \stackrel{\$}{\leftarrow} \{1, 2, \cdots, q\}$.
\end{exercise}

\begin{exercise}
Given a (strong) one-way function construct a weak one-way function that is not a (strong) one-way function.
\end{exercise}

\begin{exercise}
 Let $f:\{0,1\}^n\to \{0,1\}^{n}$ be a weak one-way permutation (a weak one way function that is a bijection). More formally, $f$ is a PPT computable one-to-one function such that $\exists$ a constant $c >0$ such that $\forall$ non-uniform PPT machine $A$ and $\forall$ sufficiently large $n$ we have that:
    \[\Pr_{x,A}[A(f(x)) \not\in f^{-1}(f(x))] > \frac{1}{n^c}\]

     Show that $g(x) = f^T(x)$ is not a strong one way permutation. Here $f^T$ denotes the $T$ times self composition of $f$ and $T$ is a polynomial in $n$.

     Interesting follow up reading if interested: With some tweaks the function above can be made a strong one-way permutation using explicit constructions of expander graphs. See Section 2.6 in \url{http://www.wisdom.weizmann.ac.il/~oded/PSBookFrag/part2N.ps}
\end{exercise}




%\subsection{Proof: Fixing a Value in a One-way Function}
%
%\begin{theorem}
%Given a one-way function $f : \binset{n} \rightarrow \binset{m}$ and constants $x_0 \in \binset{n}$, $y_0 \in \binset{m}$, $\exists g : \binset{n} \rightarrow \binset{m}$ such that $g(x_0) = y_0$ where $g$ is a one-way function.\\
%\end{theorem}
%
%Main Idea:  Set $g$ to be $f$, except at $x_0$, where $g(x_0) = y_0$.  If there exists an adversary that can break $g$, then that adversary will also break $f$, because the adversary can only know negligibly more information about $g$ than $f$.\\
%
%\proof  Define the function $g$ as follows:
%
%$g(x) = \left\{
%\begin{array}{lr}
%  y_0 & : x = x_0 \\
%  f(x) & : x \neq x_0
%\end{array}
%\right.$
%
%Suppose there is an adversary $A$ that can break $g$ with non-negligible probability $\mu(n)$.\\
%
%So, we have $\mu(n) = \underset{x \overset{\$}{\leftarrow} \binset{n}}{Pr} [ A(g(x)) \in g^{-1}(g(x)) ] = \sum\limits_{x \in \binset{n}} Pr(X = x) Pr [ A(g(x)) \in g^{-1}(g(x)) ]$\\
%
%Since $x$ is uniformly distributed, $Pr[X = x] = \frac{1}{2^n}$.  We can split it into the cases $x : g(x) = y_0$ and $x : g(x) \neq y_0$:\\
%
%$\mu(n) = \big[ \frac{1}{2^n} \sum\limits_{x \in \binset{n}, g(x) = y_0} Pr [ A(y_0) \in g^{-1}(y_0)) ] \big] + \big[ \frac{1}{2^n} \sum\limits_{x \in \binset{n}, g(x) \neq y_0} Pr [ A(g(x)) \in g^{-1}(g(x)) ] \big]$.\\
%
%Let $p = | \{ x : g(x) = y_0 \} |$.  Consider the adversary $M$ where $M(y) = x_1$ for any $y$, where $x_1$ is a value of $x$ where $f(x_1) = y_0$.  Thus, $M$ breaks $f$ for any input where $f(x) = y_0$, of which there are $p - 1$ or $p$ (depending on whether $f(x_0) = y_0$).  So, the probability with which $M$ breaks $f$ is $\frac{p-1}{2^n}$ or $\frac{p}{2^n}$.  Either way, since $f$ is a one-way function, this implies that $\frac{p}{2^n}$ is a negligible function.\\
%
%Now, since $Pr [ A(y_0) \in g^{-1}(g(x_0)) ] \leq 1$, we have:\\
%
%$\mu(n) \leq \frac{p}{2^n} + \sum\limits_{x \in \binset{n}, g(x) \neq y_0} Pr [ A(g(x)) \in g^{-1}(g(x)) ]$\\
%
%Notice that for any $x$ such that $g(x) \neq y_0$, we have $f(x) = g(x)$ and $f^{-1}(f(x)) = g^{-1}(g(x))$.\\
%
%So $\mu(n) \leq \frac{p}{2^n} + \frac{1}{2^n}\sum\limits_{x \in \binset{n}, g(x) \neq y_0} Pr [ A(f(x)) \in f^{-1}(f(x)) ]$\\
%
%Thus, if we consider $A$ as an adversary for $f$, then we get:\\
%
%$\underset{x \overset{\$}{\leftarrow} \binset{n}}{Pr} [ A(f(x)) \in f^{-1}(f(x)) ] \; \geq \; \frac{1}{2^n}\sum\limits_{x \in \binset{n}, g(x) \neq y_0} Pr [ A(f(x)) \in f^{-1}(f(x)) ] \; \geq \; \mu(n) - \frac{p}{2^n}$\\
%
%$\mu(n)$ is non-negligible and $\frac{p}{2^n}$ is negligible, and so, $\mu(n) - \frac{p}{2^n}$ is non-negligible.  Thus $A$ is an adversary that breaks $f$ with non-negligible probability.  \qed
%
