% !TEX root = collection.tex



\chapter{Public Key Encryption}
Public key encryption deals with a setting where there are two parties who wish
to communicate a secret message from one of them to the other. Unlike the symmetric
setting, in which the two parties share a secret key, the public-key setting has
asymmetry in who can decrypt a given message. This allows one party to
announce the public key to everyone so messages can be encrypted, but keep the
secret key private so no one else can preform decryption.

\begin{definition}[Public Key Encryption]

A public key crypto-system consists of three algorithms: $\gen$, $\enc$,
and $\dec$ with properties as follows:

\begin{enumerate}
\item $\gen(1^k)$ outputs a pair of keys $(\pk, \sk)$; the public and private
keys respectively.

\item $\enc(\pk, m)$ encrypts a message $m$ under
public key $\pk$.

\item $\dec(\sk, c)$ decrypts a ciphertext $c$
under secret key $\sk$.
\end{enumerate}
\end{definition}

There are other properties about these algorithms which we will discuss next
in order for these algorithms to be useful. The first of these, correctness,
ensures that the decryption of an encrypted message returns the original plaintext.
The second is the security property, which says that an attacker with access
to the encrypted message learns nothing about the plaintext.

Note that the $\gen$ algorithm must be a randomized algorithm. If not, it would
always output the same public-private key pair, and would not be very useful. We will
later show that $\enc$ must also be a randomized algorithm, or else the security
properties will not hold. Finally, $\dec$ may be a randomized algorithm but is not
required to be one.


\section{Correctness}

In order for the encryption and decryption to satisfy our intuition of what these
algorithms should do, we require that decrypting an encrypted value with the
correct keys yields the original message.

\begin{definition}[Correctness]
An public key algorithm $(\gen, \enc, \dec)$ is \emph{correct} if
\begin{equation*}
\forall m . \Pr[\dec(\sk, \enc(\pk, m))=m | (\pk,\sk) \gets \gen(1^k)] = 1
\end{equation*}
\end{definition}

Some definitions may relax this constraint from being equal to one to being greater
than $1-neg(\cdot)$. However, since this probability is equal to one, we can restate
the definition

\begin{lemma}[Correctness]
An public key algorithm $(\gen, \enc, \dec)$ is \emph{correct} if
\begin{equation*}
\forall m, \pk,\sk . (\pk,\sk) \gets \gen(1^k) \implies \dec(\sk, \enc(\pk, m))=m.
\end{equation*}
\end{lemma}

There is no statement involved about what happens when encrypting or decrypting
under the wrong key, nor is there anything about decrypting a malformed ciphertext.
It only requires that decryption of an encrypted message under corresponding
keys produces the original message.

\section{Indistinguishability and Semantic Security}

Not only must public key encryption be correct, we would also like it to hide
the values which are encrypted. There are two different definitions which we
will show are identical.

\subsection{Indistinguishability Security}
Our first definition, indistinguishability, states that the ciphertexts
obtained from encrypting any message must look identical to those from encrypting
any other message. In particular this implies that encryption must be a randomized
algorithm.

\begin{definition}[Indistinguishability]
A public key encryption scheme satisfies the indistinguishability property if
the distributions
$A_{m_1}$ and $A_{m_2}$ are computationally indistinguishable for all $m_1, m_2 \in M$
such that $|m_1| = |m_2|$ where
\begin{align*}
A_{m_1} = \{\pk, \enc(\pk, m_1) : (\pk, \sk) \gets \gen(1^k)\}_k \\
A_{m_2} = \{\pk, \enc(\pk, m_2) : (\pk, \sk) \gets \gen(1^k)\}_k
\end{align*}
\end{definition}

It is required that we talk about computational indistinguishability. The encryption
of a message must reveal at least something in an information-theoretic setting.
It is also required that the sizes of the two messages be equal. It would be very
easy to tell the difference between the encryption of a one-bit message and an
arbitrarily large message by just comparing their sizes. It is possible to work around
this requirement by having an encryption scheme pad messages so that they are all
of equal size if an upper bound is known.

\subsection{Semantic Security}
An alternate definition, which we will later prove is identical, is semantic security.
This definition intuitively states that given a ciphertext, you should learn nothing about
the original message other than it's length.

\begin{definition}[Semantic Security]
An encryption scheme is semantically secure if there exists
a simulator $S$ such that the two following processes generate computationally
indistinguishable outputs.

\begin{tabular}{cccc}
\parbox[t]{2.5cm}{\noindent \\}
&
\parbox[t]{5.8cm}{
1. $(m,z) \gets M(1^k)$ \\
2. $(\pk, \sk) \gets \gen(1^k)$ \\
3. $\text{Output\,} A(\pk, \enc(\pk, m), z)$
}
&
\parbox[t]{1cm}{
\noindent\\
$\approx$
}
&
\parbox[t]{4.5cm}{
1. $(m,z) \gets M(1^k)$ \\
2. $\text{Output\,} S^A(1^k, z)$
}
\end{tabular}
\\

Where $M$ is a machine that randomly samples message from the message space and
arbitrary additional information and $A$ is the adversary.

\end{definition}


\subsection{Equivalence of Definitions}

\begin{theorem}[Equivalence of Definitions]
A public key encryption scheme $(gen, enc, dec)$ is semantically secure if and only if
it satisfies the indistinguishability property.
\end{theorem}

\proof
We begin by proving that semantic security implies indistinguishability.
That is, we need to show that for every PPT adversary $A$, $A_{m_1}$ and $A_{m_2}$ are computationally indistinguishable for every pair of $m_1, m_2 \in M$.
We prove by a hybrid argument.
In hybrid $H_0$ let $M$ always output the value $m_1$, feed into $A$ and output $A(\pk, \enc(\pk, m_1))$.
In hybrid $H_1$ replace the output by $S^A(1^k)$.
In hybrid $H_2$ let $M$ always output the value $m_2$, feed into $A$ and output $A(\pk, \enc(\pk, m_2))$.
By semantic security the output of these hybrids are computationally indistinguishable, hence $A$ cannot distinguish $A_{m_1}$ and $A_{m_2}$.



Now we prove the other direction.
The simulator generates a public/secret key pair $(\pk, \sk) \gets \gen(1^k)$, encrypts the message $0^{|m|}$ and feeds into $A$.
It outputs $A(\pk, \enc(\pk, 0^{|m|}), z)$.
By the indistinguishability property, $A$ cannot distinguish $(\pk, \enc(\pk, 0^{|m|}))$ from $(\pk, \enc(\pk, m))$, hence the outputs of $A$ and $S$ are computationally indistinguishable.
 \qed



\section{Public Key Encryption from Trap-Door OWP}

We now show how it is possible to create a public key encryption scheme from a trapdoor
one way permutation.

\begin{theorem}
Let $(G,F,A)$ be a trap-door one-way permutation The following is then a public-key
encryption scheme:
\begin{enumerate}
\item $\gen(1^k) = (i,t_i) \gets G(1^k)$.
\item $\enc(\pk, m) = (F_i(x) || r, B(x, r) \oplus m)$ where $(x,r)$ are sampled uniformly, and $B(\cdot)$ is a hard-core bit.
\item $\dec(\sk, c) = B(A(i,t_i, y), r) \oplus c_0$ where $c = (y||r,c_0)$.
\end{enumerate}

\end{theorem}

\proof
Clearly the encryption scheme is correct, since $\alpha \oplus \alpha \oplus m = m$ for any $\alpha$.

Now we  prove  that the scheme is indistinguishable by showing that $\enc(\pk, 0)$ and $\enc(\pk, 1)$
are computationally indistinguishable. Note that distinguishing the two distributions is by definition equivalent to guessing the hard-core bit.
Hence they are indistinguishable.
\qed


\section{Indistinguishability in a Chosen Plaintext Attack}

We now define a new security requirement, IND-CPA, which is stronger than indistinguishability
or semantic security. Consider the following scheme:

The challenger samples $(\pk, \sk) \gets \gen(1^k)$  and gives $\pk$ to the attacker.
The attacker then replis with two messages $(m_0, m_1)$. The challenger then picks
one of these uniformly at random, encrypts it generating $c = \enc(\pk, m_b)$ and sends
it to the attacker. The attacker must then guess which message was encrypted.

Here, the attacker gets to choose two messages which he likes, ask the challenger to
encrypt the two messages, and only must be able to distinguish between the two.

Then a scheme is IND-CPA if $\forall A . \Pr[\text{Experiment}^{A(1^k)}] < 1/2+neg(k)$ where
A is a PPT machine. That is,
an attacker which can ask the user to encrypt specific messages after learning the
public keys can still not learn anything about the values of the encrypted messages. In
particular, there can not be any easy-to-identify weak messages for a given public key.

\begin{definition}
A public key encryption scheme (PKE) given by the three efficient procedures $(G,E,D)$ is
IND-CPA-secure if no adversary $A$ has a significant advantage in the game
represented in Table~\ref{tab:cpa}.
\begin{table}[ht]
\centering
\begin{tabular}{r c l}
Challenger & & Adversary \\
%\mright{(\pk,\sk)\gets G(1^k)}{\pk}{}
%\mleft{}{m'_0,m'_1}{}
%$b \xleftarrow{\$} \{0,1\}$ & & \\
%\mright{c^*=E(\pk,m'_b)}{c^*}{}
%\mleft{}{b'}{}
Output 1 if $(b'=b)$ & & \\
\end{tabular}
\caption{CPA security.}\label{tab:cpa}
\end{table}

This means that every probabilistic polynomial-time (PPT) adversary $A$ has only a
negligible advantage (over a random guess)
when guessing which message ($m'_0$ or $m'_1$) the challenger used to
obtain $c^*$.

Note that, since the adversary has the public key $\pk$, he is able to encrypt any
polynomial number of plaintexts during the game.
\end{definition}


\begin{theorem}
IND-CPA is a stronger definition than semantic security.
\end{theorem}

\proof
First, observe that IND-CPA is at least as strong as semantic security. An
attack which showed an encryption scheme is not semantically secure can be used
identically to distinguish the two messages the attacker sends in the IND-CPA protocol.

Now we only need to show that it is stronger.
In the following we construct an encryption scheme which is semantically secure, but is not secure
under IND-CPA.
Assume we have
a scheme $(G, E, D)$ which is semantically secure. Now we will construct a new one
$(G', E', D')$ which  is still semantically secure, but is definitely not IND-CPA.
\begin{align*}
& G'(1^k) = ((\pk, x_0, x_1), \sk), \,\text{where}\, (\pk,\sk) \gets G(1^k), x_0, x_1 \gets M(1^k) \\
& E'((\pk, x_0, x_1), m) = \text{if } m \not\in\{x_0,x_1\}, \,\text{then}\, (\mathsf{y},E(\pk,m)) \,\text{else}\, (\mathsf{n},m) \\
& D'(\sk, c) = \text{if } c \text{ starts with } \mathsf{y}, \text{then}\, D(\sk,E(\pk,m)) \,\text{else}\, m 
\end{align*}

This scheme is still semantically secure. $E'$ is different from $E$ on only two
inputs, which is certainly negligible in $k$. However, this
scheme is not IND-CPA. After seeing the public key pair $(\pk, x_0, x_1)$, the attacker
knows exactly to pick values $x_0$ and $x_1$ will be able to distinguish between
those two with probability $1$. \qed



%%%%%%%%%%%%%%%%%%%%%%%%%%%%%%%%%%%%%%%%%%%%%%%%%%%%%%%%%%%%%%%%%%%%%%%%%%%%%
\section{Chosen Ciphertext Attack for Public Key Encryption}


\begin{definition}
A public key encryption scheme (PKE) given by the three efficient procedures $(G,E,D)$ is
IND-CCA1-secure if no adversary $A$ has a significant advantage in the game
represented in Table~\ref{tab:cca1}.
\begin{table}[ht]
\centering
\begin{tabular}{r c l}
Challenger & & Adversary \\
%\mright{(\pk,\sk)\gets G(1^k)}{\pk}{}
%\mleft{}{c_1}{}
%\mright{m_1=D(\sk,c_1)}{m_1}{}
% & \vdots & \\
%\mleft{}{c_q}{}
%\mright{m_q=D(\sk,c_q)}{m_q}{}
%\mleft{}{m'_0,m'_1}{}
%$b \xleftarrow{\$} \{0,1\}$ & & \\
%\mright{c^*=E(\pk,m'_b)}{c^*}{}
%\mleft{}{b'}{}
Output 1 if $(b'=b)$ & & \\
\end{tabular}
\caption{Non-adaptive CCA security.}\label{tab:cca1}
\end{table}

In this definition the adversary may send some polynomial number of queries
to be decrypted before he receives the challenge ciphertext $c^*$.
\end{definition}



\begin{definition}
A public key encryption scheme (PKE) given by the three efficient procedures $(G,E,D)$ is
IND-CCA2-secure if no adversary $A$ has a significant advantage in the game
represented in Table~\ref{tab:cca2}.
\begin{table}[t!]
\centering
\begin{tabular}{r c l}
Challenger & & Adversary \\
%\mright{(\pk,\sk)=G(1^k)}{\pk}{}
%\mleft{}{c_1}{}
%\mright{m_1=D(\sk,c_1)}{m_1}{}
% & \vdots & \\
%\mleft{}{c_q}{}
%\mright{m_q=D(\sk,c_q)}{m_q}{}
%\mleft{}{m'_0,m'_1}{}
%$b \xleftarrow{\$} \{0,1\}$ & & \\
%\mright{c^*=E(\pk,m'_b)}{c^*}{}
%\mleft{}{c_{q+1}}{c_{q+1}\neq c^*}
%\mright{m_{q+1}=D(\sk,c_{q+1})}{m_{q+1}}{}
% & \vdots & \\
%\mleft{}{c_{q+l}}{c_{q+l}\neq c^*}
%\mright{m_{q+l}=D(\sk,c_{q+l})}{m_{q+l}}{}
%\mleft{}{b'}{}
Output 1 if $(b'=b)$ & & \\
\end{tabular}
\caption{Adaptive CCA security.}\label{tab:cca2}
\end{table}

Note that, in this adaptive version, the adversary is able to send more queries to the
challenger even after having seen the challenge ciphertext $c^*$. The only thing we require
is that he does not pass the challenge ciphertext $c^*$ itself for those queries.
\end{definition}

\begin{theorem}
Given an IND-CPA-secure public key encryption scheme $(G,E,D)$, it is possible
to construct an IND-CCA1-secure public key encryption scheme $(G',E',D')$.
\end{theorem}
\proof
Let $(\pk_1,\sk_1)\leftarrow G(1^k)$, $(\pk_2,\sk_2)\leftarrow G(1^k)$ be two pairs
of keys generated by the IND-CPA scheme $(G,E,D)$.
We claim that there is a NIZK proof system that is
able to prove that $c_1$ and $c_2$ are ciphertexts obtained by the encryption
of the same message $m$ under the keys $\pk_1$ and $\pk_2$, respectively.

More precisely, we claim that there is a NIZK proof system for the language
$$L = \{(c_1,c_2) \mid \text{$\exists r_1,r_2,m$ such that $c_1=E(\pk_1,m;r_1)$ and
$c_2=E(\pk_2,m;r_2)$}\},$$
where $E(\pk_i,m;r_i)$ represents the output of $E(\pk_i,m)$ when the random coin flips
of the procedure $E$ are given by $r_i$.

The language $L$ is clearly in $NP$, since for every $x=(c_1,c_2)\in L$ there is a witness
$w=(r_1,r_2,m)$ that proves that $x\in L$. Given $x$ and $w$, there is an efficient procedure
to verify if $w$ is a witness to the fact that $x\in L$.

By Theorem~\ref{the:NIZK_NP}, there is a NIZK proof system for any language in $NP$, so our
claim holds.
Let this NIZK proof system be given by the procedures $(K,P,V)$.
We can assume that this is an adaptive NIZK, because it is always possible to construct an
adaptive NIZK from a non-adaptive NIZK proof system.
Then we define our public key encryption scheme $(G',E',D')$ as follows:

\begin{align*}
G'(1^k):\quad & (\pk_1,\sk_1) \leftarrow G(1^k) \\
              & (\pk_2,\sk_2) \leftarrow G(1^k) \\
              & \sigma \leftarrow K(1^k) \\
              & \text{let $\pk'=(\pk_1,\pk_2,\sigma)$ and $\sk'=\sk_1$} \\
              & \text{return $(\pk',\sk')$} \\
& \\
E'(\pk',m):\quad & \text{let $(\pk_1,\pk_2,\sigma)=\pk'$} \\
                 & c_1 \leftarrow E(\pk_1,m;r_1) \\
                 & c_2 \leftarrow E(\pk_2,m;r_2) \\
                 & \text{let $x=(c_1,c_2)$ // statement to prove} \\
                 & \text{let $w=(r_1,r_2,m)$ // witness for the statement} \\
                 & \pi \leftarrow P(\sigma,x,w) \\
                 & \text{let $c=(c_1,c_2,\pi)$} \\
                 & \text{return $c$} \\
& \\
D'(\pk',c):\quad & \text{let $\sk_1=\sk'$} \\
                 & \text{let $(c_1,c_2,\pi)=c$} \\
                 & \text{let $x=(c_1,c_2)$} \\
                 & \text{if $V(\sigma,x,\pi)$} \\
                 & \quad\text{then return $D(\sk_1,c_1)$} \\
                 & \quad\text{else return $\bot$} \\
\end{align*}

The correctness of $(G',E',D')$ is easy. If the keys were generated correctly and the
messages were encrypted correctly, then $\pi$ is a valid proof for the fact that
$c_1$ and $c_2$ encrypt the same message. So the verifier $V(\sigma,x,\pi)$ will
output true and the original message $m$ will be obtained by the decryption $D(\sk_1,c_1)$.

\begin{table}[t!]
\centering
\begin{tabular}{l|l}
Game 0 &
Game 1 \\
 & \\
$(\pk_1,\sk_1),(\pk_2,\sk_2) \leftarrow G(1^k)$ &
$(\pk_1,\sk_1),(\pk_2,\sk_2) \leftarrow G(1^k)$ \\
$\sigma \leftarrow {K(1^k)}$ &
$\textcolor{red}{\sigma \leftarrow {Sim(1^k)}}$ \\
let $\pk'=(\pk_1,\pk_2,\sigma)$ and $\sk'={\sk_1}$ &
let $\pk'=(\pk_1,\pk_2,\sigma)$ and $\sk'={\sk_1}$ \\
$(m_0,m_1) \leftarrow A^{D'(\sk',\cdot)}(\pk')$ &
$(m_0,m_1) \leftarrow A^{D'(\sk',\cdot)}(\pk')$ \\
$c_1 \leftarrow E(\pk_1,{m_0};r_1), c_2 \leftarrow E(\pk_2,{m_0};r_2)$ &
$c_1 \leftarrow E(\pk_1,{m_0};r_1), c_2 \leftarrow E(\pk_2,{m_0};r_2)$ \\
$\pi \leftarrow {P(\sigma,(c_1,c_2),(r_1,r_2,{m_0}))}$ &
$\textcolor{red}{\pi \leftarrow {Sim(c_1,c_2)}}$ \\
$b \leftarrow A(\pk',c_1,c_2,\pi)$ &
$b \leftarrow A(\pk',c_1,c_2,\pi)$ \\
\hline
Game 2 &
Game 2' \\
 & \\
$(\pk_1,\sk_1),(\pk_2,\sk_2) \leftarrow G(1^k)$ &
$(\pk_1,\sk_1),(\pk_2,\sk_2) \leftarrow G(1^k)$ \\
$\sigma \leftarrow {Sim(1^k)}$ &
$\sigma \leftarrow {Sim(1^k)}$ \\
let $\pk'=(\pk_1,\pk_2,\sigma)$ and $\sk'={\sk_1}$  &
let $\pk'=(\pk_1,\pk_2,\sigma)$ and $\sk'=\textcolor{red}{\sk_2}$  \\
$(m_0,m_1) \leftarrow A^{D'(\sk',\cdot)}(\pk')$ &
$(m_0,m_1) \leftarrow A^{\textcolor{red}{D'(\sk',\cdot)}}(\pk')$ \\
$c_1 \leftarrow E(\pk_1,{m_0};r_1), c_2 \leftarrow E(\pk_2,\textcolor{red}{m_1};r_2)$ &
$c_1 \leftarrow E(\pk_1,{m_0};r_1), c_2 \leftarrow E(\pk_2,{m_1};r_2)$ \\
$\pi \leftarrow {Sim(c_1,c_2)}$ &
$\pi \leftarrow {Sim(c_1,c_2)}$ \\
$b \leftarrow A(\pk',c_1,c_2,\pi)$ &
$b \leftarrow A(\pk',c_1,c_2,\pi)$ \\
\hline
Game 3 &
Game 3' \\
 & \\
$(\pk_1,\sk_1),(\pk_2,\sk_2) \leftarrow G(1^k)$ &
$(\pk_1,\sk_1),(\pk_2,\sk_2) \leftarrow G(1^k)$ \\
$\sigma \leftarrow {Sim(1^k)}$ &
$\sigma \leftarrow {Sim(1^k)}$ \\
let $\pk'=(\pk_1,\pk_2,\sigma)$ and $\sk'={\sk_2}$  &
let $\pk'=(\pk_1,\pk_2,\sigma)$ and $\sk'=\textcolor{red}{\sk_1}$  \\
$(m_0,m_1) \leftarrow A^{D'(\sk',\cdot)}(\pk')$ &
$(m_0,m_1) \leftarrow A^{\textcolor{red}{D'(\sk',\cdot)}}(\pk')$ \\
$c_1 \leftarrow E(\pk_1,\textcolor{red}{m_1};r_1), c_2 \leftarrow E(\pk_2,{m_1};r_2)$ &
$c_1 \leftarrow E(\pk_1,{m_1};r_1), c_2 \leftarrow E(\pk_2,{m_1};r_2)$ \\
$\pi \leftarrow {Sim(c_1,c_2)}$ &
$\pi \leftarrow {Sim(c_1,c_2)}$ \\
$b \leftarrow A(\pk',c_1,c_2,\pi)$ &
$b \leftarrow A(\pk',c_1,c_2,\pi)$ \\
\hline
Game 4 & \\
 & \\
$(\pk_1,\sk_1),(\pk_2,\sk_2) \leftarrow G(1^k)$ & \\
$\sigma \leftarrow {K(1^k)}$ & \\
let $\pk'=(\pk_1,\pk_2,\sigma)$ and $\sk'={\sk_1}$  & \\
$(m_0,m_1) \leftarrow A^{D'(\sk',\cdot)}(\pk')$ & \\
$c_1 \leftarrow E(\pk_1,{m_1};r_1), c_2 \leftarrow E(\pk_2,{m_1};r_2)$ & \\
$\textcolor{red}{\pi \leftarrow {P(\sigma,(c_1,c_2),(r_1,r_2,{m_1}))}}$ & \\
$b \leftarrow A(\pk',c_1,c_2,\pi)$ & \\
\end{tabular}
\caption{Games used for the hybrid argument.}\label{tab:games}
\end{table}

Now we want to prove that $(G',E',D')$ is IND-CCA1-secure.
Let $Sim$ be a simulator of the NIZK proof system.
Consider the games shown in Table~\ref{tab:games}.
In these games, the adversary is given a decrypting oracle during the first phase
and at the end the adversary should output a bit to distinguish which game he is playing.
We now show that the adversary $A$ cannot distinguish Game 0 and Game 4.

Let $\fake$  be the event that $A$ submits a query $(c_1,c_2,\pi)$ for its decryption oracle such that $D(\sk_1,c_1) \neq D(\sk_2, c_2)$ but $V(\sigma,(c_1,c_2),\pi)=1$.
This represents the event where the adversary is able to trick the verifier into returning true on a bad pair of ciphertexts (i.e., a pair not in the language).
The probability of event $\fake$ in Game 0 is negligible because of the
soundness of the proof system.

Game 1 differs from Game 0 by the use of a simulator for the random string and proof generation.
They are indistinguishable by reduction to the zero-knowledge property of the proof system.
In particular, if the adversary could detect the difference, then the adversary could be used to distinguish real proofs from simulated proofs.
And the probability of event $\fake$ in Game 1 is also negligible by the
zero-knowledge property.

Game 2 differs from Game 1 in that it obtains ciphertext $c_2$ from the message $m_1$.
They are indistinguishable by reduction to the IND-CPA property of the underlying PKE,
which guarantees that encryptions of $m_0$ cannot be distinguished from encryptions of $m_1$.
$\Pr[\fake]$ is the same as in Game 1, namely negligible.

Game 2' differs from Game 2 by using the key $\sk_2$ for decryption instead of $\sk_1$.
The adversary's view of the two games only differs if the event $\fake$ occurs, and it happens with
negligible probability.

Game 3 differs from Game 2' in that it obtains ciphertext $c_1$ from the message $m_1$.
They are indistinguishable by reduction to the IND-CPA property of the underlying PKE,
which guarantees that encryptions of $m_0$ cannot be distinguished from encryptions of $m_1$.
$\Pr[\fake]$ stays the same, namely negligible.

Game 3' differs from Game 3 by using the key $\sk_1$ for decryption instead of $\sk_2$.
The adversary's view of the two games only differs if the event $\fake$ occurs, and it happens with
negligible probability.

Game 4 differs from Game 3' by the use of the real NIZK proof system instead of a simulator.
They are indistinguishable by reduction to the zero-knowledge property of the proof system.
\qed


