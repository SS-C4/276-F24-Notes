
%%%%%%%%%%%%%%%%%%%%%%%%%%%%%%%%%%%%%%%%%%%%%%%%%%%%%
\section{Hardness Amplification}
\label{sec:owf:amplify}
In this section, we show that even a very \emph{weak} form of one-way functions suffices from constructing one-way functions as defined previously. For this section, we refer to this previously defined notion as strong one-way functions.
\begin{definition}[Weak One-Way Functions]
A function $f : \binset{n} \rightarrow \binset{m}$ is said to be a weak one-way function if:
\begin{itemize}
\item[-] $f$ is computable by a polynomial time machine, and
\item[-] There exists a noticeable function $\alpha_f(\cdot)$ such that $\forall$ non-uniform PPT adversaries $\mathcal{A}$ we have that
    $$
    \mu_{\mathcal{A},f}(n) =
    \Pr_{x \stackrel{\$}{\leftarrow} \binset{n}}[ \mathcal{A}(1^n, f(x)) \in f^{-1}(f(x))] \leq 1 - \alpha_{f}(n).
    $$
\end{itemize}
\end{definition}

\begin{theorem}\label{theorem:weakstrongOWF}
If there exists a weak one-way function, then there exists a (strong) one-way function.
\end{theorem}

\proof We prove the above theorem constructively. Suppose $f : \binset{n} \rightarrow \binset{m}$ is a weak one-way function, then we prove that the function $g: \binset{nq} \rightarrow \binset{mq}$ for $q = \lceil \frac{2n}{\alpha_{f}(n)} \rceil$ where 
$$g(x_1, x_2, \cdots, x_q) = f(x_1) || f(x_2) || \cdots || f(x_q),$$
 is a strong one-way function.

Assume for the sake of contradiction that there exists an adversary $\mathcal{B}$ such that $\mu_{\mathcal{B},g}(nq) = \Pr_{x \stackrel{\$}{\leftarrow} \binset{nq}}[ \mathcal{B}(1^{nq}, g(x)) \in g^{-1}(g(x))]$ is non-negligible.
%Suppose $\mu_{\mathcal{A},g}(nq) \geq \tilde \mu_{\mathcal{A},g}(nq)$ for arbitrarily large $n$, where $\tilde  \mu_{\mathcal{A},g}$ is a noticeable function.\peihan{to ensure that $T$ is poly}
Then we use $\mathcal{B}$ to construct $\mathcal{A}$ (see Figure~\ref{fig:adv:weak}) that breaks $f$, namely $\mu_{\mathcal{A},f}(n) = \Pr_{x \stackrel{\$}{\leftarrow} \binset{n}}[ \mathcal{A}(1^n, f(x)) \in f^{-1}(f(x))] > 1 - \alpha_f(n)$ for sufficiently large $n$.
\begin{marginfigure}[-5cm]
%\Loop { $T=\frac{4n^2}{\alpha_f(n) \mu_{\mathcal{B}, g}(nq)}$ times}
\begin{enumerate}
    \item $i \stackrel{\$}{\leftarrow} [q]$.
    \item $x_1, \cdots, x_{i-1}, x_i, \cdots, x_q \stackrel{\$}{\leftarrow} \binset{n}$.
    \item Set $y_j = f(x_j)$ for each $j \in [q]\backslash \{i\}$ and $y_i = y$.
    \item $(x'_1, x'_2, \cdots, x'_q) := \mathcal{B} (f(x_1), f(x_2), \cdots, f(x_q))$.
    \item {$f(x'_i) = y$} then output $x'_i$ else $\bot$.
\end{enumerate}
\caption{Construction of $\mathcal{A}(1^n, y)$}
\label{fig:adv:weak}
\end{marginfigure}

Note that: (1) $\mathcal{A}(1^n, y)$ iterates at most $T = \frac{4n^2}{\alpha_f(n)\mu_{\mathcal{B},g}(nq)}$ times each call is polynomial time. (2) $\mu_{\mathcal{B},g}(nq)$ is a non-negligible function. This implies that for infinite choices of $n$ this value is greater than some noticeable function. Together these two facts imply that for infinite choices of $n$ the running time of $\mathcal{A}$ is bounded by a polynomial function in $n$.

It remains to show that $\Pr_{x \stackrel{\$}{\leftarrow} \binset{n}}[ \mathcal{A}(1^n, f(x)) = \bot] < \alpha_f(n)$ for arbitrarily large $n$. A natural way to argue this is by showing that at least one execution of $\mathcal{B}$ should suffice for inverting $f(x)$. However, the technical challenge in proving this formally is that these calls to $\mathcal{B}$ aren't independent. Below we formalize this argument even when these calls aren't independent.\marginnote[-5cm]{\begin{lemma}
Let $A$ be any an efficient algorithm such that $\Pr_{x,r}[A(x,r) =1] \geq \epsilon$. Additionally, let $G = \{x\mid \geq \Pr_{r}[A(x,r) =1] \geq \frac\epsilon2\}$. Then, we have $\Pr_x[x \in G] \geq \frac\epsilon2$.
\end{lemma}
\begin{proof}
The proof of this lemma follows by a very simple counting argument. Let's start by assuming that $\Pr_x[x \in G] < \frac\epsilon2$. Next, observe that
\begin{align*}
\Pr_{x,r}&[A(x,r) =1]& \\&= \Pr_x[x \in G]\cdot\Pr_{x,r}[A(x,r) =1\mid x \in G] \\&+ \Pr_x[x \not\in G]\cdot\Pr_{x,r}[A(x,r) =1\mid x \not\in G]
\\&< \frac\epsilon2\cdot 1 + 1\cdot\frac\epsilon2
\\&< \epsilon,
\end{align*}
which is a contradiction.
\end{proof}
}

Define the set $S$ of ``bad'' $x$'s, which are hard to invert:
$$S := \left\{x \left| \Pr_\mathcal{B}\left[\mathcal{A} \text{ inverts $f(x)$ in a single iteration} \right] \leq \frac{\alpha_f(n) \mu_{\mathcal{B},g}(nq)}{4n} \right. \right\}.$$
We start by proving that the size of $S$ is small. More formally,
$$\Pr_{x \stackrel{\$}{\leftarrow} \binset{n}} [x \in S] \leq \frac{\alpha_f(n)}{2}.$$
Assume, for the sake of contradiction,\marginnote{\begin{lemma}
Let $A$ be any an efficient algorithm such that $\Pr_{x,r}[A(x_1,\ldots x_n,r) =1] \geq \epsilon$. Additionally, let $G = \{x\mid \geq \Pr_{x_1,\ldots x_n,r}[A(x,r) =1\mid \exists i, x = x_i] \geq \frac\epsilon2\}$. Then, we have $\Pr_x[x \in G] \geq \frac\epsilon2$.
\end{lemma}
\begin{proof}
The proof of this lemma follows by a very simple counting argument. Let's start by assuming that $\Pr_x[x \in G] < \frac\epsilon2$. Next, observe that
\begin{align*}
\Pr_{x,r}&[A(x,r) =1]& \\&= \Pr_x[x \in G]\cdot\Pr_{x,r}[A(x,r) =1\mid x \in G] \\&+ \Pr_x[x \not\in G]\cdot\Pr_{x,r}[A(x,r) =1\mid x \not\in G]
\\&< \frac\epsilon2\cdot 1 + 1\cdot\frac\epsilon2
\\&< \epsilon,
\end{align*}
which is a contradiction.
\end{proof}
}
that $\Pr_{x \stackrel{\$}{\leftarrow} \binset{n}} [x \in S]  > \frac{\alpha_f(n)}{2}$. Then we have that:
\begin{align*}
\mu_{\mathcal{B},g}(nq) =& \Pr_{(x_1, \cdots, x_q) \stackrel{\$}{\leftarrow} \binset{nq}}[ \mathcal{B}(1^{nq}, g(x_1, \cdots, x_q)) \in g^{-1}(g(x_1, \cdots, x_q))]\\
=&  \Pr_{x_1, \cdots, x_q}[ \mathcal{B}(1^{nq}, g(x_1, \cdots, x_q)) \in g^{-1}(g(x_1, \cdots, x_q)) \wedge \forall i: x_i \notin S]\\
& + \Pr_{x_1, \cdots, x_q}[ \mathcal{B}(1^{nq}, g(x_1, \cdots, x_q)) \in g^{-1}(g(x_1, \cdots, x_q)) \wedge \exists i: x_i \in S]\\
\leq& \Pr_{x_1, \cdots, x_q}[ \forall i: x_i \notin S]
+ \sum_{i=1}^q \Pr_{x_1, \cdots, x_q}[ \mathcal{B}(1^{nq}, g(x_1, \cdots, x_q)) \in g^{-1}(g(x_1, \cdots, x_q)) \wedge  x_i \in S]\\
\leq& \left( 1-\frac{\alpha_f(n)}{2}\right)^q
+ q \cdot \Pr_{x_1, \cdots, x_q,i}[ \mathcal{B}(1^{nq}, g(x_1, \cdots, x_q)) \in g^{-1}(g(x_1, \cdots, x_q)) \wedge x_i \in S] \\
=& \left( 1-\frac{\alpha_f(n)}{2}\right)^{\frac{2n}{\alpha_f(n)}}
+  q\cdot \Pr_{x \stackrel{\$}{\leftarrow} \binset{n}, \mathcal{B}}[\mathcal{A} \text{ inverts $f(x)$ in a single iteration}  \wedge x \in S]\\
\leq& e^{-n} + q\cdot  \Pr_{x}[x \in S] \cdot \Pr[\mathcal{A} \text{ inverts $f(x)$ in a single iteration} ~|~ x \in S]\\
\leq& e^{-n} + \frac{2n}{\alpha_f(n)} \cdot  1 \cdot \frac{\mu_{\mathcal{B},g}(nq) \cdot \alpha_f(n)}{4n}\\
\leq& e^{-n} + \frac{\mu_{\mathcal{B},g}(nq)}{2}.
\end{align*}
Hence $\mu_{\mathcal{B},g}(nq) \leq 2 e^{-n}$, contradicting with the fact that $\mu_{\mathcal{B},g}$ is non-negligible.
Then we have
\begin{align*}
\Pr_{x \stackrel{\$}{\leftarrow} \binset{n}}&[ \mathcal{A}(1^n, f(x)) = \bot]\\
=& \Pr_x[x \in S] + \Pr_x [x \notin S]\cdot\Pr[\mathcal{B} \text{ fails to invert $f(x)$ in every iteration} | x \notin S]\\
\leq& \frac{\alpha_f(n)}{2}+ \left(\Pr[ \mathcal{B} \text{ fails to invert $f(x)$ a single iteration} | x \notin S] \right)^T\\
\leq & \frac{\alpha_f(n)}{2}+ \left( 1-\frac{\mu_{\mathcal{A},g}(nq) \cdot \alpha_f(n)}{4n}\right)^T\\
\leq& \frac{\alpha_f(n)}{2} + e^{-n} \leq \alpha_f(n)
\end{align*}
for sufficiently large $n$. This concludes the proof.
\qed


\section{Levin's One-Way Function}
\begin{theorem}\label{thm:levin}
If there exists a one-way function, then there exists an explicit function $f$ that is one-way  (constructively).
\end{theorem}

\begin{lemma}\label{lem:n2owf}
If there exists a one-way function computable in time $n^c$ for a constant $c$, then there exists a one-way function computable in time $n^2$.
\end{lemma}
\proof
Let $f: \binset{n} \rightarrow \binset{n}$ be a one-way function computable in time $n^c$.
Construct $g: \binset{n+n^c} \rightarrow \binset{n+n^c}$ as follows:
$$g(x,y) = f(x) || y$$
where $x \in \binset{n}, y \in \binset{n^c}$.
$g(x,y)$ takes time $2n^c$, which is linear in the input length.

We next show that $g(\cdot)$ is one-way.
Assume for the purpose of contradiction that there exists an adversary $\mathcal{A}$ such that $\mu_{\mathcal{A},g}(n+n^c) = \Pr_{(x,y) \stackrel{\$}{\leftarrow} \binset{n+n^c}}[ \mathcal{A}(1^{n+n^c}, g(x,y)) \in g^{-1}(g(x,y))]$ is non-negligible. Then we use $\mathcal{A}$ to construct $\mathcal{B}$ such that $\mu_{\mathcal{B},f}(n) = \Pr_{x \stackrel{\$}{\leftarrow} \binset{n}}[ \mathcal{B}(1^n, f(x)) \in f^{-1}(f(x))]$ is also non-negligible.

$\mathcal{B}$ on input $z \in\{0,1\}^n$, samples $y \stackrel{\$}{\leftarrow} \binset{n^c}$, and outputs the $n$ higher-order bits of  $\mathcal{A}(1^{n+n^c}, z||y)$. Then we have
\begin{align*}
\mu_{\mathcal{B},g}(n) =& \Pr_{x \stackrel{\$}{\leftarrow} \binset{n}, y \stackrel{\$}{\leftarrow} \binset{n^c}}\left[\mathcal{A}(1^{n+n^c}, f(x) || y) \in f^{-1}(f(x)) || \binset{n^c}\right]\\
\geq&\Pr_{x,y}\left[\mathcal{A}(1^{n+n^c}, g(x,y)) \in f^{-1}(f(x)) || y\right]\\
=& \Pr_{x,y}\left[\mathcal{A}(1^{n+n^c}, g(x,y)) \in g^{-1}(g(x,y))\right]
\end{align*}
is non-negligible.
\qed

\bigskip
\proof[Proof of Theorem~\ref{thm:levin}]
We first construct a weak one-way function $h: \binset{n} \rightarrow \binset{n}$ as follows:
$$
h(M,x) = \left\{
\begin{array}{ll}
M || M(x) & \text{if $M(x)$ takes no more than $|x|^2$ steps} \\
M || 0 & \text{otherwise}
\end{array}
\right.
$$
where $|M| = \log n, |x| = n - \log n$ (interpreting $M$ as the code of a machine  and $x$ as its input).
If $h$ is weak one-way, then we can construct a one-way function from $h$ as we discussed in Section~\ref{sec:owf:amplify}.

It remains to show that if one-way functions exist, then $h$ is a weak one-way function, with $\alpha_h(n) = \frac{1}{n^2}$.
Assume for the purpose of contradiction that there exists an adversary $\mathcal{A}$ such that $\mu_{\mathcal{A},h}(n) = \Pr_{(M,x) \stackrel{\$}{\leftarrow} \binset{n}}[ \mathcal{A}(1^{n}, h(M,x)) \in h^{-1}(h(M,x))]\geq 1-\frac{1}{n^2}$ for all sufficiently large $n$.
By the existence of one-way functions and Lemma~\ref{lem:n2owf}, there exists a one-way function $\tilde M$ that can be computed in time $n^2$. Let $\tilde M$ be the uniform machine that computes this one-way function.
We will consider values $n$ such that $n > 2^{|\tilde M|}$. In other words for these choices of $n$, $\tilde M$ can be described using $\log n$ bits.
We construct $\mathcal{B}$ to invert $\tilde M$: on input $y$ outputs the $(n-\log n)$ lower-order bits of $\mathcal{A}(1^n, \tilde M||y)$. Then
\begin{align*}
\mu_{\mathcal{B},\tilde M}(n-\log n) =& \Pr_{x \stackrel{\$}{\leftarrow} \binset{n-\log n}}\left[\mathcal{A}(1^{n}, \tilde M || \tilde M(x)) \in \binset{\log n} || \tilde M^{-1}(\tilde M((x))\right]\\
\geq& \Pr_{x \stackrel{\$}{\leftarrow} \binset{n-\log n}}\left[\mathcal{A}(1^{n}, \tilde M || \tilde M(x)) \in \tilde{M} || \tilde M^{-1}(\tilde M((x))\right].
\end{align*}
Observe that for sufficiently large $n$ it holds that
\begin{align*}
1-\frac{1}{n^2} \leq& \mu_{\mathcal{A},h}(n)\\
=& \Pr_{(M,x) \stackrel{\$}{\leftarrow} \binset{n}}\left[ \mathcal{A}(1^{n}, h(M,x)) \in h^{-1}(h(M,x))\right]\\
\leq& \Pr_{M }[M = \tilde M] \cdot \Pr_{x }\left[ \mathcal{A}(1^{n}, \tilde M||\tilde M(x)) \in  \tilde{M} || \tilde M^{-1}(\tilde M((x))\right] + \Pr_{M }[M \neq \tilde M]  \\
\leq&  \frac{1}{n} \cdot \mu_{\mathcal{B},\tilde M}(n-\log n) +\frac{n-1}{n}.
\end{align*}
Hence $\mu_{\mathcal{B},\tilde M}(n-\log n) \geq \frac{n-1}{n}$  for sufficiently large $n$ which is a contradiction.
\qed

