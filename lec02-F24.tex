
%%%%%%%%%%%%%%%%%%%%%%%%%%%%%%%%%%%%%%%%%%%%%%%%%%%%%
\section{Hardness Amplification}
\label{sec:owf:amplify}
In this section, we show that even a very \emph{weak} form of one-way functions suffices from constructing one-way functions as defined previously. For this section, we refer to this previously defined notion as strong one-way functions.
\begin{definition}[Weak One-Way Functions]
A function $f : \binset{n} \rightarrow \binset{m}$ is said to be a weak one-way function if:
\begin{itemize}
\item[-] $f$ is computable by a polynomial time machine, and
\item[-] There exists a noticeable function $\alpha_f(\cdot)$ such that $\forall$ non-uniform PPT adversaries $\mathcal{A}$ we have that
    $$
    \mu_{\mathcal{A},f}(n) =
    \Pr_{x \stackrel{\$}{\leftarrow} \binset{n}}[ \mathcal{A}(1^n, f(x)) \in f^{-1}(f(x))] \leq 1 - \alpha_{f}(n).
    $$
\end{itemize}
\end{definition}

\begin{theorem}\label{theorem:weakstrongOWF}
If there exists a weak one-way function, then there exists a (strong) one-way function.
\end{theorem}

\proof We prove the above theorem constructively. Suppose $f : \binset{n} \rightarrow \binset{m}$ is a weak one-way function, then we prove that the function $g: \binset{nq} \rightarrow \binset{mq}$ for $q = \lceil \frac{2n}{\alpha_{f}(n)} \rceil$ where 
$$g(x_1, x_2, \cdots, x_q) = f(x_1) || f(x_2) || \cdots || f(x_q),$$
 is a strong one-way function. Let us discuss the intuition. A weak one-way function is "strong" in a small part of its domain. For this construction to result in a strong one-way function, we need just one of the $q$ instantiations to be in the part of the domain where our weak one-way function is strong. If we pick a large enough $q$, this is guaranteed to happen.

Assume for the sake of contradiction that there exists an adversary $\mathcal{B}$ such that $\mu_{\mathcal{B},g}(nq) = \Pr_{x \stackrel{\$}{\leftarrow} \binset{nq}}[ \mathcal{B}(1^{nq}, g(x)) \in g^{-1}(g(x))]$ is non-negligible.
%Suppose $\mu_{\mathcal{A},g}(nq) \geq \tilde \mu_{\mathcal{A},g}(nq)$ for arbitrarily large $n$, where $\tilde  \mu_{\mathcal{A},g}$ is a noticeable function.\peihan{to ensure that $T$ is poly}
Then we use $\mathcal{B}$ to construct $\mathcal{A}$ (see Figure~\ref{fig:adv:weak}) that breaks $f$, namely $\mu_{\mathcal{A},f}(n) = \Pr_{x \stackrel{\$}{\leftarrow} \binset{n}}[ \mathcal{A}(1^n, f(x)) \in f^{-1}(f(x))] > 1 - \alpha_f(n)$ for sufficiently large $n$.
\begin{marginfigure}[-5cm]
%\Loop { $T=\frac{4n^2}{\alpha_f(n) \mu_{\mathcal{B}, g}(nq)}$ times}
\begin{enumerate}
    \item $i \stackrel{\$}{\leftarrow} [q]$.
    \item $x_1, \cdots, x_{i-1}, x_i, \cdots, x_q \stackrel{\$}{\leftarrow} \binset{n}$.
    \item Set $y_j = f(x_j)$ for each $j \in [q]\backslash \{i\}$ and $y_i = y$.
    \item $(x'_1, x'_2, \cdots, x'_q) := \mathcal{B} (f(x_1), f(x_2), \cdots, f(x_q))$.
    \item {$f(x'_i) = y$} then output $x'_i$ else $\bot$.
\end{enumerate}
\caption{Construction of $\mathcal{A}(1^n, y)$}
\label{fig:adv:weak}
\end{marginfigure}

Note that: (1) $\mathcal{A}(1^n, y)$ iterates at most $T = \frac{4n^2}{\alpha_f(n)\mu_{\mathcal{B},g}(nq)}$ times each call is polynomial time. (2) $\mu_{\mathcal{B},g}(nq)$ is a non-negligible function. This implies that for infinite choices of $n$ this value is greater than some noticeable function. Together these two facts imply that for infinite choices of $n$ the running time of $\mathcal{A}$ is bounded by a polynomial function in $n$.

It remains to show that $\Pr_{x \stackrel{\$}{\leftarrow} \binset{n}}[ \mathcal{A}(1^n, f(x)) = \bot] < \alpha_f(n)$ for arbitrarily large $n$. A natural way to argue this is by showing that at least one execution of $\mathcal{B}$ should suffice for inverting $f(x)$. However, the technical challenge in proving this formally is that these calls to $\mathcal{B}$ aren't independent. Below we formalize this argument even when these calls aren't independent.\marginnote[-5cm]{\begin{lemma}
Let $A$ be any an efficient algorithm such that $\Pr_{x,r}[A(x,r) =1] \geq \epsilon$. Additionally, let $G = \{x\mid \geq \Pr_{r}[A(x,r) =1] \geq \frac\epsilon2\}$. Then, we have $\Pr_x[x \in G] \geq \frac\epsilon2$.
\end{lemma}
\begin{proof}
The proof of this lemma follows by a very simple counting argument. Let's start by assuming that $\Pr_x[x \in G] < \frac\epsilon2$. Next, observe that
\begin{align*}
\Pr_{x,r}&[A(x,r) =1]& \\&= \Pr_x[x \in G]\cdot\Pr_{x,r}[A(x,r) =1\mid x \in G] \\&+ \Pr_x[x \not\in G]\cdot\Pr_{x,r}[A(x,r) =1\mid x \not\in G]
\\&< \frac\epsilon2\cdot 1 + 1\cdot\frac\epsilon2
\\&< \epsilon,
\end{align*}
which is a contradiction.
\end{proof}
}

Define the set $S$ of ``bad'' $x$'s, which are hard to invert:
$$S := \left\{x \left| \Pr_\mathcal{B}\left[\mathcal{A} \text{ inverts $f(x)$ in a single iteration} \right] \leq \frac{\alpha_f(n) \mu_{\mathcal{B},g}(nq)}{4n} \right. \right\}.$$
We start by proving that the size of $S$ is small. More formally,
$$\Pr_{x \stackrel{\$}{\leftarrow} \binset{n}} [x \in S] \leq \frac{\alpha_f(n)}{2}.$$
Assume, for the sake of contradiction,\marginnote{\begin{lemma}
Let $A$ be any an efficient algorithm such that $\Pr_{x,r}[A(x_1,\ldots x_n,r) =1] \geq \epsilon$. Additionally, let $G = \{x\mid \geq \Pr_{x_1,\ldots x_n,r}[A(x,r) =1\mid \exists i, x = x_i] \geq \frac\epsilon2\}$. Then, we have $\Pr_x[x \in G] \geq \frac\epsilon2$.
\end{lemma}
\begin{proof}
The proof of this lemma follows by a very simple counting argument. Let's start by assuming that $\Pr_x[x \in G] < \frac\epsilon2$. Next, observe that
\begin{align*}
\Pr_{x,r}&[A(x,r) =1]& \\&= \Pr_x[x \in G]\cdot\Pr_{x,r}[A(x,r) =1\mid x \in G] \\&+ \Pr_x[x \not\in G]\cdot\Pr_{x,r}[A(x,r) =1\mid x \not\in G]
\\&< \frac\epsilon2\cdot 1 + 1\cdot\frac\epsilon2
\\&< \epsilon,
\end{align*}
which is a contradiction.
\end{proof}
}
that $\Pr_{x \stackrel{\$}{\leftarrow} \binset{n}} [x \in S]  > \frac{\alpha_f(n)}{2}$. Then we have that:
\begin{align*}
\mu_{\mathcal{B},g}(nq) =& \Pr_{(x_1, \cdots, x_q) \stackrel{\$}{\leftarrow} \binset{nq}}[ \mathcal{B}(1^{nq}, g(x_1, \cdots, x_q)) \in g^{-1}(g(x_1, \cdots, x_q))]\\
=&  \Pr_{x_1, \cdots, x_q}[ \mathcal{B}(1^{nq}, g(x_1, \cdots, x_q)) \in g^{-1}(g(x_1, \cdots, x_q)) \wedge \forall i: x_i \notin S]\\
& + \Pr_{x_1, \cdots, x_q}[ \mathcal{B}(1^{nq}, g(x_1, \cdots, x_q)) \in g^{-1}(g(x_1, \cdots, x_q)) \wedge \exists i: x_i \in S]\\
\leq& \Pr_{x_1, \cdots, x_q}[ \forall i: x_i \notin S]
+ \sum_{i=1}^q \Pr_{x_1, \cdots, x_q}[ \mathcal{B}(1^{nq}, g(x_1, \cdots, x_q)) \in g^{-1}(g(x_1, \cdots, x_q)) \wedge  x_i \in S]\\
\leq& \left( 1-\frac{\alpha_f(n)}{2}\right)^q
+ q \cdot \Pr_{x_1, \cdots, x_q,i}[ \mathcal{B}(1^{nq}, g(x_1, \cdots, x_q)) \in g^{-1}(g(x_1, \cdots, x_q)) \wedge x_i \in S] \\
=& \left( 1-\frac{\alpha_f(n)}{2}\right)^{\frac{2n}{\alpha_f(n)}}
+  q\cdot \Pr_{x \stackrel{\$}{\leftarrow} \binset{n}, \mathcal{B}}[\mathcal{A} \text{ inverts $f(x)$ in a single iteration}  \wedge x \in S]\\
\leq& e^{-n} + q\cdot  \Pr_{x}[x \in S] \cdot \Pr[\mathcal{A} \text{ inverts $f(x)$ in a single iteration} ~|~ x \in S]\\
\leq& e^{-n} + \frac{2n}{\alpha_f(n)} \cdot  1 \cdot \frac{\mu_{\mathcal{B},g}(nq) \cdot \alpha_f(n)}{4n}\\
\leq& e^{-n} + \frac{\mu_{\mathcal{B},g}(nq)}{2}.
\end{align*}
Hence $\mu_{\mathcal{B},g}(nq) \leq 2 e^{-n}$, contradicting with the fact that $\mu_{\mathcal{B},g}$ is non-negligible.
Then we have
\begin{align*}
\Pr_{x \stackrel{\$}{\leftarrow} \binset{n}}&[ \mathcal{A}(1^n, f(x)) = \bot]\\
=& \Pr_x[x \in S] + \Pr_x [x \notin S]\cdot\Pr[\mathcal{B} \text{ fails to invert $f(x)$ in every iteration} | x \notin S]\\
\leq& \frac{\alpha_f(n)}{2}+ \left(\Pr[ \mathcal{B} \text{ fails to invert $f(x)$ a single iteration} | x \notin S] \right)^T\\
\leq & \frac{\alpha_f(n)}{2}+ \left( 1-\frac{\mu_{\mathcal{A},g}(nq) \cdot \alpha_f(n)}{4n}\right)^T\\
\leq& \frac{\alpha_f(n)}{2} + e^{-n} \leq \alpha_f(n)
\end{align*}
for sufficiently large $n$. This concludes the proof.
\qed
