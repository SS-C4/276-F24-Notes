We show how to use the arbitrary-length WOOF we constructed to boost this one-time, fixed-length digital signature scheme into a one-time, arbitrary-length digital signature scheme. 

\subsection{Removing Length-Restriction from One-Time Digital Signatures} 

Let $(\Gen_\ell, \Sign_\ell, \Verify_\ell)$ be a length-restricted one-time digital signature for messages of length $\ell(n)$. 
Let $h_s: \{0,1\}^* \to \{0,1\}^n$ be a WOOF. 
First we will review an \emph{insecure} first attempt at a construction for $\ell(n) = n$: 
\begin{itemize}
    \item $\Gen^{\text{BAD}}(1^n)$: Run $(pk_\ell, sk_\ell) \leftarrow \Gen_\ell(1^n)$, $s \leftarrow I(1^n)$. Output $((pk_\ell,s),sk_\ell)$.

    \item $\Sign^{\text{BAD}}(sk=sk_\ell, m)$: Output $\Sign_\ell(sk_\ell, h_s(m))$. 

    \item $\Verify^{\text{BAD}}(pk=(pk_\ell,s), m, \sigma)$: Output $\Verify_\ell(pk_\ell, h_s(m), \sigma)$. 
\end{itemize}
Notice that this construction does not work because the seed for the WOOF is revealed before the message is chosen, in which case WOOF security does not apply. 

To avoid this, we can use the following construction for $\ell(n) = k(n) + n$ where $k(n)$ is the length of the seed produced by $I(1^n)$: 
\begin{itemize}
    \item $\Gen(1^n)$: Run $(pk_\ell, sk_\ell) \leftarrow \Gen_\ell(1^n)$. Output $(pk_\ell,sk_\ell)$.

    \item $\Sign(sk=sk_\ell, m)$: Run $s \leftarrow I(1^n)$ and $\sigma_\ell \leftarrow \Sign_\ell(sk_\ell, s || h_s(m))$. Output $(\sigma_\ell, s)$. 

    \item $\Verify(pk=pk_\ell, m, \sigma = (\sigma_\ell, s))$: Output $\Verify_\ell(pk_\ell, s || h_s(m), \sigma_\ell)$. 
\end{itemize}
Note that now the seed is chosen after the message. 
We will give a proof sketch of the security for this construction. 
Assume for contradiction that we have a nu-PPT adversary $\adv$ which succeeds with non-negligible probability at the digital signature security game for $(\Gen, \Sign, \Verify)$. 
We will now construct a nu-PPT adversary $\advb$ who succeeds with non-negligible probability at the digital signature security game for $(\Gen_\ell, \Sign_\ell, \Verify_\ell)$, a contradiction. 
$\advb$ receives $pk_\ell$ from its challenger and passes this along to $\adv$. 
When $\adv$ queries the signing oracle with message $m$, $\advb$ runs $s \leftarrow I(1^n)$, computes $m_{\ell} = s || h_{s}(m)$, queries its signing oracle with message $m_{\ell}$ to receive $\sigma_\ell$ and returns $\sigma = (s, \sigma_{\ell})$ to $\adv$. 
Finally, when $\adv$ returns $m^*, \sigma^* = (s^*, \sigma_{\ell}^*)$, $\advb$ outputs $m^*_{\ell} = s^* || h_{s^*}(m^*), \sigma_{\ell}^*$. 

Since we have replicated the expected input distribution for $\adv$, it will succeed with non-negligible probability. 
Notice that $\advb$ will succeed when $\adv$ does as long as $m^*_{\ell} \not = m_{\ell}$. 
In analyzing the success probability of $\advb$ we have two cases to consider based on whether $s^* = s$. 
Notice that $\adv$ must have non-negligible success either when $s^* = s$ or when $s^* \not = s$ or both. 
If $\Pr[s^* \not = s \land \adv \text{ succeeds}]$ is non-negligible, then $\advb$ also succeeds with non-negligible probability since $m^*_{\ell} = s^* || h_{s^*}(m^*) \not = s || h_{s}(m) = m_{\ell}$ in this case. 
Now assume that $\Pr[s^* = s \land \adv \text{ succeeds}]$ is non-negligible. When $\adv$ succeeds in this case, it must have found $m^* \not = m$, and so $h_{s^*}(m^*) = h_{s}(m)$ with only negligible probability because otherwise WOOF security is broken. 
Thus with non-negligible probability $m^*_{\ell} \not = m_{\ell}$ and $\advb$ succeeds as well. 
Therefore either way $\advb$ succeeds in the digital signature game against $(\Gen_\ell, \Sign_\ell, \Verify_\ell)$ with non-negligible probability. 

\section{Multiple-Message Digital Signatures} 

Now we will show how to covert this one-time, no-length restriction digital signature scheme $(\Gen, \Sign, \Verify)$ into a ef-ema no-length restriction digital signature scheme by utilizing a pseudorandom function $PRF$. 
For each $\alpha \in \{\epsilon\} \cup \{0,1\}^{\leq n}$, let $pk_\alpha, sk_\alpha = \Gen(1^n; PRF_s(\alpha 10 \ldots 0))$ such that $|\alpha 10 \ldots 0| = n+1$ (i.e. $\Gen$ is run with randomness determined by the PRF on an input specified by $\alpha$). 
We will use these to make a tree of keys so that the keys used for each message will be distinct with high probability, so WOOF security will continue to apply each time the scheme is used. 
Note that whenever $\Sign$ is called, we require the WOOF to be run with a deterministic seed $s_{\alpha}^W = PRF_{s'}(\alpha 10 \ldots 0)$. 
This way paths through the tree will deterministically map to the corresponding signatures. 
The construction is as follows: 
\begin{itemize}
    \item $\mathsf{GEN}(1^n)$: Output $(pk_\epsilon, s \leftarrow \{0,1\}^n)$, namely the root public key and the seed for the PRF so the rest of the keys can be generated. 

    \item $\mathsf{SIGN}(sk, m)$: 
    \begin{enumerate}
        \item Draw a random path through the key tree $r \leftarrow \{0,1\}^n$. 

        \item Now use the secret key at each level to sign its children's public keys and continue to do this along the random path until a leaf is hit, i.e. iteratively sign the random path and its co-path. 
        Namely, for each $i = 0, 1, \ldots, n-1$, let $\alpha_i = r_1 r_2 \cdots r_i$, $m_i = pk_{\alpha_i || 0} || pk_{\alpha_i || 1}$, and $\sigma_i = \Sign(sk_{\alpha_i}, m_i)$. 

        \item Let $\sigma_n = \Sign(sk_r, m)$. 

        \item Output $\Sigma = (r, m_0, \sigma_0, \ldots, m_{n-1}, \sigma_{n-1}, \sigma_n)$. 
    \end{enumerate} 

    \item $\mathsf{VERIFY}(pk, m, \Sigma = (r, \sigma_0, \ldots, \sigma_{n-1}, \sigma_n))$: Use $r$ and the $m_i$ to determine $pk_{\alpha_i}$ and for each $i \in [n]$ run $\Verify(pk_{\alpha_i}, m_i, \sigma_i)$, accepting if all of those do. 
\end{itemize}
The idea is that because the root $pk_{\epsilon}$ is trusted and the corresponding secret keys of one level are used to validate the $pk$ of the level below, trust is maintained down the path and the ultimate $pk_r$ can be trusted to be used to check the signature on $m$ itself. 

We will now give a proof sketch for the security of this construction. 
Assume for contradiction that we have a nu-PPT adversary $\adv$ which succeeds with non-negligible probability at the digital signature security game for $(\mathsf{GEN}, \mathsf{SIGN}, \mathsf{VERIFY})$. 
Let $\adv$'s interaction with this security game be called Hybrid $H_0$. 
First we will consider the hybrid $H_1$ where the PRF is replaced by a truly random function. 
$\adv$'s success probability in $H_1$ is still some non-negligible $\epsilon(n)$ due to PRF security. 
Now we will consider the hybrid $H_2$ where if the randomness returned by any two of the signing oracle queries is equal, i.e. $r_j = r_{j'}$ for some distinct $j, j' \in [q]$, then $\adv$ aborts. 
Notice that the probability of this happening is only $q^2/2^n$, a negligible amount, so $\adv$'s success probability $\epsilon(n) - q^2/2^n$ remains non-negligible. 
Thus going forward we can assume that the randomness $r_j$ used by the oracle to sign each query $m_j$ is distinct. 

Now notice that either $\adv$ outputs a message-signature pair $(M^*, \Sigma^*)$ which uses $r^* = r_j$ for some $j \in [q]$ or $r^* \not = r_j$ for all $j \in [q]$. 
We will give some intuition for what happens in each of these cases. 
In the first case, since signatures in this scheme are deterministic, to succeed $\adv$'s signatures must be the same as $\Sigma_j$'s along the path $r^* = r_j$ until doing a forgery at the leaf, breaking the $(\Gen, \Sign, \Verify)$ scheme for $M^*$ which is distinct from all of the $m_j$ queries. 
In the second case, $\adv$ goes along a path $r^* \not = r_j$, so at the first node which diverges from all $r_j$ it must forge a signature that verifies with an honest public key from the level above, breaking the $(\Gen, \Sign, \Verify)$ scheme for the corresponding $\alpha_i^*$ which is distinct from all of the other $\alpha_i^j$ that were used to answer queries. 

This can be formalized by constructing a nu-PPT adversary $\advb$ for the one-time digital signature security game of $(\Gen, \Sign, \Verify)$ who takes the $pk$ it's been given and guesses which query $\adv$ forges for, using its $pk$ and one-time oracle to provide a signature for that query and otherwise answering honestly using self-generated keys. 
This degrades $\advb$'s probability of success by $\epsilon(n)/q$ which is still non-negligible. 

Next, after introducing collision resistant hash functions, we will see a different, though closely related, alternate construction for multiple-message digital signatures. 


\section{Collision Resistant Hash Functions}

As the name suggests, collision resistant hash function family is a set of hash functions $H$ such that for a function $h$ chosen randomly from the family, it is computationally hard to find two different inputs $x,x'$ such that $h(x) = h(x')$. We now give a formal definition.

\subsection{Definition of a family of CRHF}

A set of function ensembles
\[ \{H_n = \{h_i : D_n \to R_n \}_{i \in I_n} \}_n\]
where $|D_n| < |R_n|$ is a family of collision resistant hash function ensemble if there exists efficient algorithms $(\Sampler,\Eval)$ with the following syntax:
\begin{enumerate}
\item $\Sampler(1^n) \to i:$ On input $1^n$, $\Sampler$ outputs an index $i \in I_n$.
\item $\Eval(i,x) = h_i(x):$ On input $i$ and $x \in D_n$, $\Eval$ algorithm outputs $h_i(x)$. 
\item $\forall$ PPT $\adv$ we have
\[\Pr[i \gets \Sampler(1^n), (x,x') \gets \adv(1^n,i) : h_i(x) = h_i(x') \wedge x \neq x'] \leq \negl(n)\]
\end{enumerate}


\subsection{Collision Resistant Hash functions from Discrete Log}
We will now give a construction of collision resistant hash functions from the discrete log assumption. We first recall the discrete log assumption:
\begin{definition}[Discrete-Log Assumption]
We say that the discrete-log assumption holds for the group ensemble $\mathcal{G} =\{ \mathbb{G}_n\}_{n \in \mathbb{N}}$, if for every non-uniform PPT algorithm $\mathcal{A}$ we have that
\[\mu_\mathcal{A}(n) := \Pr_{x \leftarrow |G_n|}[\mathcal{A}(g,g^x) = x]\]
is a negligible function.
\end{definition}

We now give a construction of collision resistant hash functions.  

\begin{itemize}
\item $\Sampler(1^n):$ On input $1^n$, the sampler does the following:
\begin{enumerate}
    \item It chooses $x \gets |\mathbb{G}_n|$.
    \item It computes $h = g^x$.
    \item It outputs $(g,h)$.

\end{enumerate}
\item $\Eval((g,h),(r,s)):$ On input $(g,h)$ and two elements $(r,s) \in |\mathbb{G}_n|$, $\Eval$ outputs $g^rh^s$.
\end{itemize}

We now argue that this construction is collision resistant. Assume for the sake of contradiction that an adversary gives a collision $(r_1,s_1) \neq (r_2,s_2)$. We will now use this to compute the discrete logarithm of $h$. We first observe that:
\begin{eqnarray*}
r_1+xs_1 &=& r_2 + xs_2\\
(r_1 - r_2) &=& x(s_2 - s_1)
\end{eqnarray*}
We infer that $s_2 \neq s_1$. Otherwise, we get that $r_1 = r_2$ and hence, $(r_1,s_1) = (r_2,s_2)$. Thus, we can compute $x = \frac{r_1-r_2}{s_1 - s_2}$ and hence the discrete logarithm of $h$ is computable.


\section{CRHF-Based Multiple-Message Digital Signature}

We now explain how to combine collision-resistant hash functions and one-time signatures to get a signature scheme for multiple messages. We first construct an intermediate primitive wherein we will still have the same security property as that of one-time signature but we would be able to sign messages longer than the length of the public-key.\footnote{Note that in the one-time signature scheme that we constructed earlier, the length of message that can be signed is same as the length of the public-key.}


\subsection{One-time Signature Scheme for Long Messages}
We first observe that the CRHF family $H$ that we constructed earlier compresses $2n$ bits to $n$ bits (also called as 2-1 CRHF). We will now give an extension that compresses an arbitrary long string to $n$ bits using a 2-1 CRHF.
\paragraph{Merkle-Damgard CRHF.} The sampler for this CRHF is same as that of 2-1 CRHF. Let $h$ be the sampled hash function. To hash a string $x$, we do the following. Let $x$ be a string of length $m$ where $m$ is an arbitrary polynomial in $n$. We will assume that $m = kn$ (for some $k$) or otherwise, we can pad $x$ to this length. We will partition  the string $x$ into $k$ blocks of length $n$ each. For simplicity, we will assume that $k$ is a perfect power of $2$ or we will again pad $x$ appropriately. We will view these $k$-blocks as the leaves of a complete binary tree of depth $\ell = \log_2 k$. Each intermediate node is associated with a bit string $y$ of length at most $\ell$ and the root is associated with the empty string. We will assign a $\tag \in \bin^n$ to each node in the tree. The $i$-th leaf is assigned $\tag_i$ equal to the $i$-block of the string $x$. Each intermediate node $y$ is assigned a $\tag_y = h(\tag_{y\|0}\| \tag_{y \| 1})$. The output of the hash function is set to be the $\tag$ value of the root. Notice that if there is a collision for this CRHF then there are exists one intermediate node $y$ such that for two different values $\tag_{y\|0},\tag_{y\|1}$ and $\tag'_{y\|0},\tag'_{y\|1}$ we have, $h(\tag_{y\|0},\tag_{y\|1}) = \tag'_{y\|0},\tag'_{y\|1}$. This implies that there is a collision for $h$. 

\paragraph{Construction.} We will now use the Merkle-Damgard CRHF and the one-time signature scheme that we constructed earlier to get a one-time signature scheme for signing longer messages. The main idea is simple: we will sample a $(sk,vk)$ for signing $n$-bit messages and to sign a longer message, we will first hash it using the Merkle-Damgard hash function to $n$-bits and then sign on the hash value. The security of the construction follows directly from the security of the one-time signature scheme since the CRHF is collision-resistant. 

\subsection{Signature Scheme for Multiple Messages}
We will now describe the construction of signature scheme for multiple messages. Let $(\Gen',\Sign',\Verify')$ be a one-time signature scheme for signing longer messages. 
\begin{enumerate}
    \item $\Gen(1^n):$ Run $\Gen'(1^n)$ using to obtain $sk,vk$. Sample a PRF key $K$. The signing key is $(sk,K)$ and the verification key is $vk$.
    \item $\Sign((sk,K),m):$ To sign a message $m$, do the following:
    \begin{enumerate}
        \item Parse $m$ as $m_1m_2\ldots m_{\ell}$ where each $m_i \in \bin$.
        \item Set $sk_0 = sk$ and $m_0 = \epsilon$ (where $\epsilon$ is the empty string).
        \item For each $i \in [\ell]$ do:
        \begin{enumerate}
            \item Evaluate $\PRF(m_1\|\ldots\|m_{i-1}\|0)$ and $\PRF(m_1\|\ldots\|m_{i-1}\|1)$ to obtain $r_0$ and $r_1$ respectively. Run $\Gen'(1^n)$ using $r_0$ and $r_1$ as the randomness to obtain $(sk_{i,0},vk_{i,1})$ and $(sk_{i,1},vk_{i,1})$.
            \item Set $\sigma_i = \Sign(sk_{i-1,m_{i-1}},vk_{i,0}\|vk_{i,1})$
            \item If $i = \ell$, then set $\sigma_{\ell+1} = \Sign(sk_{i,m_i},m)$.
            
        \end{enumerate}
        \item Output $\sigma = (\sigma_1,\ldots,\sigma_{\ell+1})$ along with all the verification keys as the signature.
    \end{enumerate}
    \item $\Verify(vk,\sigma,m)$: Check if all the signatures in $\sigma$ are valid.
\end{enumerate}

To prove security, we will first use the security of the PRF to replace the outputs with random strings. We will then use the security of the one-time signature scheme to argue that the adversary cannot mount an existential forgery.

\section{Trapdoor Permutations and RSA} 

\begin{definition}[Trapdoor Permutation] 
A function family $\{f_s: D_s \to D_s\}_{s \in \{0,1\}^*}$ is a \emph{one-way trapdoor permutation} if there exists PPT $I, D, F, F^{-1}$ such that 
\begin{itemize}
    \item $(s, \tau) \leftarrow I(1^n)$ produces the seed and trapdoor, 
    
    \item $D(s)$ outputs a uniformly random element of $D_s$, 

    \item $\forall s \in I(1^n), x \in D_s$, $F(s, x) = f_s(x)$, 

    \item $\forall (s, \tau) \in I(1^n), y \in D_s$, $F^{-1}(\tau, y) = f_s^{-1}(x)$, and 

    \item $f_s$ is one-way. 
\end{itemize}
\end{definition} 

The RSA trapdoor permutation construction is as follows: 
\begin{itemize} 
    
    \item $I_{RSA}(1^n) \rightarrow (s = (N, e), \tau = (N, d)$ for $N = PQ$ for $2^{n-1} \leq P < Q \leq 2^n$ such that $d = e^{-1} \mod \phi(n)$ for $e < N$ which is coprime to $\phi(n) = (P-1)(Q-1)$. 
    Let $D_s = \{1, \ldots, N\}$. 

    \item $F_{RSA}(s, x) = x^e \mod N$. 

    \item $F_{RSA}^{-1}(\tau, x) = y^d \mod N$. 
\end{itemize}
Unfortunately, under the assumption that factoring is hard, we still don't have a security proof for the RSA trapdoor permutation in the plain model. 
However, we will see a proof in what's called the ``random oracle model'' next. 



\section*{Exercises}
\begin{exercise}
\textbf{Digital signature schemes can be made deterministic.} Given a digital signature scheme $(\mathsf{Gen}, \mathsf{Sign}, \mathsf{Verify})$ for which $\mathsf{Sign}$ is probabilistic, provide a construction of a digital signature scheme $(\mathsf{Gen}', \mathsf{Sign}', \mathsf{Verify}')$ where $\mathsf{Sign}'$ is deterministic.
\end{exercise}
