% !TEX root = collection.tex
\newcommand{\Gen}{\mathsf{Gen}}
\newcommand{\Sign}{\mathsf{Sign}}
\newcommand{\Verify}{\mathsf{Verify}}
\newcommand{\negl}{\mathsf{negl}}
\newcommand{\abort}{\mathsf{abort}}
\newcommand{\Sampler}{\mathsf{Sampler}}
\newcommand{\Eval}{\mathsf{Eval}}
\renewcommand{\tag}{\mathsf{tag}}
\newcommand{\PRF}{\mathsf{PRF}}

\chapter{Digital Signatures}

In this chapter, we will introduce the notion of a digital signature. At an intuitive level, a digital signature scheme helps providing authenticity of messages and ensuring non-repudiation. We will first define this primitive and then construct what is called as one-time secure digital signature scheme. An one-time digital signature satisfies a weaker security property when compared to digital signatures. We then introduce the concept of collision-resistant hash functions and then use this along with a one-time secure digital signature to give a construction of digital signature scheme.

\section{Definition}

A digital signature scheme is a tuple of three algorithms $(\Gen,\Sign,\Verify)$ with the following syntax:
\begin{enumerate}
\item $\Gen(1^n)\to (vk,sk)$: On input the message length (in unary) $1^n$, $\Gen$ outputs a secret signing key $sk$ and a public verification key $vk$.
\item $\Sign(sk, m) \to \sigma$: On input a secret key $sk$ and a message $m$ of length $n$, the $\Sign$ algorithm outputs a signature $\sigma$.
\item $\Verify(vk, m, \sigma) \to \{0,1\}$: On input the verification key $vk$, a message $m$ and a signature $\sigma$, the $\Verify$ algorithm outputs either $0$ or $1$.
\end{enumerate}

We require that the digital signature to satisfy the following correctness and security properties.\\
\medskip
\noindent\textbf{Correctness.} For the correctness of the scheme, we have that
$\forall m \in \bin^n$,
\[\Pr \left[ (vk,sk) \gets \Gen(1^n), \sigma \leftarrow \Sign(sk,m) : \Verify(vk, m, \sigma) = 1 \right] = 1.\]

\medskip
\noindent\textbf{Security.} Consider the following game between an adversary and a challenger
.

\begin{enumerate}
    \item The challenger first samples $(vk,sk) \gets \Gen(1^n)$. The challenger gives $vk$ to the adversary.
    \item \textbf{Signing Oracle.} The adversary is now given access to a signing oracle. When the adversary gives a query $m$ to the oracle, it gets back $\sigma \gets \Sign(sk,m)$.
    \item \textbf{Forgery.} The adversary outputs a message, signature pair $(m^*,\sigma^*)$ where $m^*$ is different from the queries that adversary has made to the signing oracle.
    \item The adversary wins the game if $\Verify(vk,m^*,\sigma^*) = 1$.
\end{enumerate}
We say that the digital signature scheme is secure if the probability that the adversary wins the game is $\negl(n)$.

\section{One-time Digital Signature}
\label{lampart}
An one-time digital signature has the same syntax and correctness requirement as that of a digital signature scheme except that in the security game the adversary is allowed to call the signing oracle only once (hence the name one-time). We will now give a construction of one-time signature scheme from the assumption that one-way functions exists.

Let $f: \bin^n \rightarrow \bin^n$ be a one-way function.
\begin{itemize}
\item $\Gen(1^n)$: On input the message length (in unary) $1^n$, $\Gen$ does the following:
\begin{enumerate}
    \item Chooses $x_{i,b} \gets \bin^n$ for each $i \in [n]$ and $b \in \bin$.
    \item Output $vk = \left[ \begin{array}{ccc}
f(x_{1,0}) & \ldots & f(x_{n,0}) \\
f(x_{1,1}) & \ldots & f(x_{n,1}) \\
\end{array} \right]$ and $sk = \left[ \begin{array}{ccc}
x_{1,0} & \ldots & x_{n,0} \\
x_{1,1} & \ldots & x_{n,1} \\
\end{array} \right]$
\end{enumerate}
\item $\Sign(sk, m)$: On input a secret key $sk$ and a message $m \in \bin^n$, the $\Sign$ algorithm outputs a signature $\sigma = x_{1,m_1}\|x_{2,m_2}\| \ldots \| x_{n,m_n}$.
\item $\Verify(vk, m, \sigma)$: On input the verification key $vk$, a message $m$ and a signature $\sigma$, the $\Verify$ algorithm does the following:
\begin{enumerate}
    \item Parse $\sigma = x_{1,m_1}\|x_{2,m_2}\| \ldots \| x_{n,m_n}$.
    \item Compute $vk'_{i,m_i} = f(x_{i,m_i})$ for each $i \in [n]$.
    \item Check if for each $i \in [n]$, $vk'_{i,m_i} = vk_{i,m_i}$. If all the checks pass, output 1. Else, output 0.
\end{enumerate}
\end{itemize}

Before we prove any security property, we first observe that this scheme is completely broken if we allow the adversary to ask for two signatures. This is because the adversary can query for the signatures on $0^n$ and $1^n$ respectively and the adversary gets the entire secret key. The adversary can then use this secret key to sign on any message and break the security. 

We will now argue the one-time security of this construction. Let $\adv$ be an adversary who breaks the security of our one-time digital signature scheme with non-negligible probability $\mu(n)$. We will now construct an adversary $\advb$ that breaks the one-wayness of $f$. $\advb$ receives a one-way function challenge $y$ and does the following:
\begin{enumerate}
    \item $\advb$ chooses $i^*$ uniformly at random from $[n]$ and $b^*$ uniformly at random from $\bin$.
    \item It sets $vk_{i^*,b^*} = y$
    \item For all $i \in [n]$ and $b \in \bin$ such that $(i,b) \neq (i^*,b^*)$, $\advb$ samples $x_{i,b} \gets \bin^n$. It computes $vk_{i,b} = f(x_{i,b})$.
    \item It sets $vk = \left[ \begin{array}{ccc}
vk_{1,0} & \ldots& vk_{n,0} \\
vk_{1,1} & \ldots& vk_{n,1} \\
\end{array} \right]$ and sends $vk$ to $\adv$.
\item $\adv$ now asks for a signing query on a message $m$. If $m_{i^*} = b^*$ then $\advb$ aborts and outputs a special symbol $\abort_1$. Otherwise, it uses it knowledge of $x_{i,b}$ for $(i,b) \neq (i^*,b^*)$ to output a signature on $m$.
\item $\adv$ outputs a valid forgery $(m^*,\sigma^*)$. If $m^*_{i^*} = m_{i^*}$ then $\advb$ aborts and outputs a special symbol $\abort_2$. If it does not abort, then it parses $\sigma^*$ as ${1,m_1}\|x_{2,m_2}\| \ldots \| x_{n,m_n}$ and outputs $x_{i^*,b^*}$ as the inverse of $y$.
\end{enumerate}
We first note that conditioned on $\advb$ not outputting $\abort_1$ or $\abort_2$, the probability that $\advb$ outputs a valid preimage of $y$ is $\mu(n)$. Now, probability $\advb$ does not output $\abort_1$ or $\abort_2$ is $1/2n$ (this is because $\abort_1$ is not output with probability $1/2$ and conditioned on not outputting $\abort_1$, $\abort_2$ is not output with probability $1/n$). Thus, $\advb$ outputs a valid preimage with probability $\mu(n)/2n$. This completes the proof of security.

We now try to extend this one-time signature scheme to digital signatures. For this purpose, we will rely on a primitive called as collision-resistant hash functions.

\section{Collision Resistant Hash Functions}

As the name suggests, collision resistant hash function family is a set of hash functions $H$ such that for a function $h$ chosen randomly from the family, it is computationally hard to find two different inputs $x,x'$ such that $h(x) = h(x')$. We now give a formal definition.

\subsection{Definition of a family of CRHF}

A set of function ensembles
\[ \{H_n = \{h_i : D_n \to R_n \}_{i \in I_n} \}_n\]
where $|D_n| < |R_n|$ is a family of collision resistant hash function ensemble if there exists efficient algorithms $(\Sampler,\Eval)$ with the following syntax:
\begin{enumerate}
\item $\Sampler(1^n) \to i:$ On input $1^n$, $\Sampler$ outputs an index $i \in I_n$.
\item $\Eval(i,x) = h_i(x):$ On input $i$ and $x \in D_n$, $\Eval$ algorithm outputs $h_i(x)$. 
\item $\forall$ PPT $\adv$ we have
\[\Pr[i \gets \Sampler(1^n), (x,x') \gets \adv(1^n,i) : h_i(x) = h_i(x') \wedge x \neq x'] \leq \negl(n)\]
\end{enumerate}


\subsection{Collision Resistant Hash functions from Discrete Log}
We will now give a construction of collision resistant hash functions from the discrete log assumption. We first recall the discrete log assumption:
\begin{definition}[Discrete-Log Assumption]
We say that the discrete-log assumption holds for the group ensemble $\mathcal{G} =\{ \mathbb{G}_n\}_{n \in \mathbb{N}}$, if for every non-uniform PPT algorithm $\mathcal{A}$ we have that
\[\mu_\mathcal{A}(n) := \Pr_{x \leftarrow |G_n|}[\mathcal{A}(g,g^x) = x]\]
is a negligible function.
\end{definition}

We now give a construction of collision resistant hash functions.  

\begin{itemize}
\item $\Sampler(1^n):$ On input $1^n$, the sampler does the following:
\begin{enumerate}
    \item It chooses $x \gets |\mathbb{G}_n|$.
    \item It computes $h = g^x$.
    \item It outputs $(g,h)$.

\end{enumerate}
\item $\Eval((g,h),(r,s)):$ On input $(g,h)$ and two elements $(r,s) \in |\mathbb{G}_n|$, $\Eval$ outputs $g^rh^s$.
\end{itemize}

We now argue that this construction is collision resistant. Assume for the sake of contradiction that an adversary gives a collision $(r_1,s_1) \neq (r_2,s_2)$. We will now use this to compute the discrete logarithm of $h$. We first observe that:
\begin{eqnarray*}
r_1+xs_1 &=& r_2 + xs_2\\
(r_1 - r_2) &=& x(s_2 - s_1)
\end{eqnarray*}
We infer that $s_2 \neq s_1$. Otherwise, we get that $r_1 = r_2$ and hence, $(r_1,s_1) = (r_2,s_2)$. Thus, we can compute $x = \frac{r_1-r_2}{s_1 - s_2}$ and hence the discrete logarithm of $h$ is computable.


\section{Multiple-Message Digital Signature}

We now explain how to combine collision-resistant hash functions and one-time signatures to get a signature scheme for multiple messages. We first construct an intermediate primitive wherein we will still have the same security property as that of one-time signature but we would be able to sign messages longer than the length of the public-key.\footnote{Note that in the one-time signature scheme that we constructed earlier, the length of message that can be signed is same as the length of the public-key.}


\subsection{One-time Signature Scheme for Long Messages}
We first observe that the CRHF family $H$ that we constructed earlier compresses $2n$ bits to $n$ bits (also called as 2-1 CRHF). We will now give an extension that compresses an arbitrary long string to $n$ bits using a 2-1 CRHF.
\paragraph{Merkle-Damgard CRHF.} The sampler for this CRHF is same as that of 2-1 CRHF. Let $h$ be the sampled hash function. To hash a string $x$, we do the following. Let $x$ be a string of length $m$ where $m$ is an arbitrary polynomial in $n$. We will assume that $m = kn$ (for some $k$) or otherwise, we can pad $x$ to this length. We will partition  the string $x$ into $k$ blocks of length $n$ each. For simplicity, we will assume that $k$ is a perfect power of $2$ or we will again pad $x$ appropriately. We will view these $k$-blocks as the leaves of a complete binary tree of depth $\ell = \log_2 k$. Each intermediate node is associated with a bit string $y$ of length at most $\ell$ and the root is associated with the empty string. We will assign a $\tag \in \bin^n$ to each node in the tree. The $i$-th leaf is assigned $\tag_i$ equal to the $i$-block of the string $x$. Each intermediate node $y$ is assigned a $\tag_y = h(\tag_{y\|0}\| \tag_{y \| 1})$. The output of the hash function is set to be the $\tag$ value of the root. Notice that if there is a collision for this CRHF then there are exists one intermediate node $y$ such that for two different values $\tag_{y\|0},\tag_{y\|1}$ and $\tag'_{y\|0},\tag'_{y\|1}$ we have, $h(\tag_{y\|0},\tag_{y\|1}) = \tag'_{y\|0},\tag'_{y\|1}$. This implies that there is a collision for $h$. 

\paragraph{Construction.} We will now use the Merkle-Damgard CRHF and the one-time signature scheme that we constructed earlier to get a one-time signature scheme for signing longer messages. The main idea is simple: we will sample a $(sk,vk)$ for signing $n$-bit messages and to sign a longer message, we will first hash it using the Merkle-Damgard hash function to $n$-bits and then sign on the hash value. The security of the construction follows directly from the security of the one-time signature scheme since the CRHF is collision-resistant. 

\subsection{Signature Scheme for Multiple Messages}
We will now describe the construction of signature scheme for multiple messages. Let $(\Gen',\Sign',\Verify')$ be a one-time signature scheme for signing longer messages. 
\begin{enumerate}
    \item $\Gen(1^n):$ Run $\Gen'(1^n)$ using to obtain $sk,vk$. Sample a PRF key $K$. The signing key is $(sk,K)$ and the verification key is $vk$.
    \item $\Sign((sk,K),m):$ To sign a message $m$, do the following:
    \begin{enumerate}
        \item Parse $m$ as $m_1m_2\ldots m_{\ell}$ where each $m_i \in \bin$.
        \item Set $sk_0 = sk$ and $m_0 = \epsilon$ (where $\epsilon$ is the empty string).
        \item For each $i \in [\ell]$ do:
        \begin{enumerate}
            \item Evaluate $\PRF(m_1\|\ldots\|m_{i-1}\|0)$ and $\PRF(m_1\|\ldots\|m_{i-1}\|1)$ to obtain $r_0$ and $r_1$ respectively. Run $\Gen'(1^n)$ using $r_0$ and $r_1$ as the randomness to obtain $(sk_{i,0},vk_{i,1})$ and $(sk_{i,1},vk_{i,1})$.
            \item Set $\sigma_i = \Sign(sk_{i-1,m_{i-1}},vk_{i,0}\|vk_{i,1})$
            \item If $i = \ell$, then set $\sigma_{\ell+1} = \Sign(sk_{i,m_i},m)$.
            
        \end{enumerate}
        \item Output $\sigma = (\sigma_1,\ldots,\sigma_{\ell+1})$ along with all the verification keys as the signature.
    \end{enumerate}
    \item $\Verify(vk,\sigma,m)$: Check if all the signatures in $\sigma$ are valid.
\end{enumerate}

To prove security, we will first use the security of the PRF to replace the outputs with random strings. We will then use the security of the one-time signature scheme to argue that the adversary cannot mount an existential forgery.

\section*{Exercises}
\begin{exercise}
\textbf{Digital signature schemes can be made deterministic.} Given a digital signature scheme $(\mathsf{Gen}, \mathsf{Sign}, \mathsf{Verify})$ for which $\mathsf{Sign}$ is probabilistic, provide a construction of a digital signature scheme $(\mathsf{Gen}', \mathsf{Sign}', \mathsf{Verify}')$ where $\mathsf{Sign}'$ is deterministic.
\end{exercise}
